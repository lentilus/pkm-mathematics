%! TeX root = zettel.tex
\documentclass[class=article, crop=false]{standalone}
\usepackage[subpreambles=true]{standalone}
% basics
\usepackage[utf8]{inputenc}
\usepackage[ngerman]{babel}
\usepackage{amsmath,amssymb,amsfonts,amsthm}
\usepackage{faktor}
\usepackage{thmtools}
\usepackage{mathtools} % prettier math
\usepackage{mathdots}
\usepackage{enumitem}
\usepackage{comment}
\usepackage{etoolbox}

\usepackage{currfile}
\usepackage{subfiles}

% frames
\usepackage[linewidth=1pt]{mdframed}
\usepackage{framed}
\usepackage{tcolorbox}
\definecolor{shadecolor}{rgb}{0.9,0.9,0.9}

% graphics
\usepackage{import}
\usepackage{xifthen}
\usepackage{pdfpages}
\usepackage{transparent}

% amsthm config
\declaretheoremstyle[notebraces={[}{]},headpunct=\newline,]{custom}
\theoremstyle{custom}
\newtheorem*{theorem}{Theorem}
\newtheorem*{lemma}{Lemma}
\newtheorem*{corollary}{Corollary}

\theoremstyle{custom}
\newtheorem*{axiom}{Axiom}
\newtheorem*{definition}{Definition}
\newtheorem*{example}{Example}
\newtheorem*{remark}{Remark}

% symbol shortcuts
\newcommand{\zz}{\mathrm{Z\kern-.4em\raise-0.5ex\hbox{Z}}}
\newcommand{\tx}[1]{\text{ #1 }}
\newcommand{\from}{\colon}
\newcommand{\La}{\mathcal{L}}
\newcommand{\N}{\mathbb{N}}
\newcommand{\K}{\mathbb{K}}
\newcommand{\R}{\mathbb{R}}
\newcommand{\Q}{\mathbb{Q}}
\newcommand{\C}{\mathbb{C}}

% math operators
\DeclareMathOperator{\Mat}{Mat}
\DeclareMathOperator{\sgn}{sgn}
\DeclareMathOperator{\Eig}{Eig}
\DeclareMathOperator{\Image}{Im}
\DeclareMathOperator{\Hom}{Hom}
\DeclareMathOperator{\End}{End}
\DeclareMathOperator{\GL}{GL}
\DeclareMathOperator{\HP}{HP}
\DeclareMathOperator{\rand}{rand}
\DeclareMathOperator{\ord}{ord}

% misc
\setlength\parindent{0pt}

% quantifiers
\let\oldforall\forall
\let\oldexists\exists
\renewcommand{\forall}{\ \oldforall}
\renewcommand{\exists}{\ \oldexists}

\makeatletter
    % nice way to write sets
    \newcommand{\set}[1]{\@ifnextchar\bgroup {\left\{#1\filteredset} {\{#1\}} }
    \newcommand{\filteredset}[1]{\ \setseperator \ #1 \right\}}
    \newcommand{\setseperator}{\middle|}
\makeatother
\newcommand{\norm}[1]{\left|\left|#1\right|\right|}

\newenvironment{flashcard}{}{}
\newenvironment{question}{\paragraph{Question}}{\vspace{5pt}}
\newenvironment{zettel}[1]
{
    \begin{mdframed}[%
        nobreak=true,%
        topline=false,%
        bottomline=false,%
        rightline=false%
        ]
    \section*{#1}
}
{
    \end{mdframed}
}
\newcommand{\inkfig}[1]{%
    % we cant use figure here because of boxes...
    \centering
    \def\svgwidth{\columnwidth}
    \import{\currfiledir figures}{#1.pdf_tex}
}


\begin{document}
\begin{zettel}{Lebesgue-Integrierbarkeit}

\begin{definition}
	\[
		f_+:= \max \set{f(x), 0}, \quad  f_-:= \min \set{0, f(x)}
	.\]
\end{definition}

\begin{flashcard}[cvdhrfg9]{Lebesgue-Integrierbarkeit, l-Integral}
	$\Omega \in  \lmable$  und $f: \Omega \to  \hat{\R  }$  $f$ ist integrierbar falls
	\[
		\int_{\Omega} |f | \leq \infty
	.\]
	\[
		\int_{\Omega} f := \int_{\Omega} f_+ - \int_{\Omega} f_-
	.\]
\end{flashcard}

\begin{flashcard}[ziotrt7v]{Lebesgue-Integrierbarkeit (aus Riemann)}
	$\Omega \in  \lmable, f:\Omega \to  \R $ beliebig.

	\begin{align*}
		\bar{\int_{\Omega}f}       & := \inf \set{\int_{\Omega} g}{g:\Omega \to  \R \tx{int-bar}, g(x)\geq  f(x)\forall x \in  \Omega } \\
		\underline{\int_{\Omega}f} & := \inf \set{\int_{\Omega} g}{g:\Omega \to  \R \tx{int-bar}, g(x)\leq  f(x)\forall x \in  \Omega } \\
	\end{align*}

	Bemerke, dass $\bar{\int_{\Omega} f} \geq \underline{\int_{\Omega}f}$

	\begin{lemma}
		$f$ genau dann integrierbar wenn
		\[
			\bar{\int_{\Omega}f} = \underline{\int_{\Omega}f} \qquad \left(= \int_{\Omega} f \right)
		.\]
	\end{lemma}
\end{flashcard}

\begin{lemma}[Eigenschaften l-integrierbarer Funktionen]
	% TODO
\end{lemma}

\begin{flashcard}[cqtcrghu]{Lebesgue-Integrierbarkeit $\R ^1$ }
	\begin{theorem}
		$R \supset I$ beschr"ankt und $f: I \to  \R $ r-integrierbar $\implies$ $f$ l-integrierbar (und Integrale gleich.  )
	\end{theorem}
\end{flashcard}

\end{zettel}
\end{document}

