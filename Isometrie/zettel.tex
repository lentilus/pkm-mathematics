%! TeX root = zettel.tex
\documentclass[class=article, crop=false]{standalone}
\usepackage[subpreambles=true]{standalone}
% basics
\usepackage[utf8]{inputenc}
\usepackage[ngerman]{babel}
\usepackage{amsmath,amssymb,amsfonts,amsthm}
\usepackage{thmtools}
\usepackage{mathtools} % prettier math
\usepackage{mathdots}
\usepackage{enumitem}
\usepackage{comment}

% frames
\usepackage[linewidth=1pt]{mdframed}
\usepackage{framed}
\usepackage{tcolorbox}
\definecolor{shadecolor}{rgb}{0.9,0.9,0.9}

% graphics
\usepackage{import}
\usepackage{xifthen}
\usepackage{pdfpages}

% amsthm config
\declaretheoremstyle[notebraces={[}{]},headpunct= ,]{custom}

\theoremstyle{custom}
\newtheorem*{theorem}{Theorem}
\newtheorem*{lemma}{Lemma}
\newtheorem*{corollary}{Corollary}

\theoremstyle{custom}
\newtheorem*{axiom}{Axiom}
\newtheorem*{definition}{Definition}
\newtheorem*{example}{Example}

% symbol shortcuts
\newcommand{\zz}{\mathrm{Z\kern-.4em\raise-0.5ex\hbox{Z}}}
\newcommand{\tx}[1]{\text{ #1 }}
\newcommand{\La}{\mathcal{L}}
\newcommand{\N}{\mathbb{N}}
\newcommand{\K}{\mathbb{K}}
\newcommand{\R}{\mathbb{R}}
\newcommand{\Q}{\mathbb{Q}}

% math operators
\DeclareMathOperator{\Mat}{Mat}
\DeclareMathOperator{\sgn}{sgn}
\DeclareMathOperator{\Eig}{Eig}
\DeclareMathOperator{\Image}{Im}
\DeclareMathOperator{\Hom}{Hom}
\DeclareMathOperator{\End}{End}
\DeclareMathOperator{\GL}{GL}

% misc
\setlength\parindent{0pt}

\makeatletter

% nice way to write sets
\newcommand{\set}[1]{\@ifnextchar\bgroup {\left\{#1\setwithcondition} {\{#1\}} }
\newcommand{\setwithcondition}[1]{\ \setseperator \ #1 \right\}}
\newcommand{\setseperator}{\middle|}

\makeatother

\newenvironment{zettel}[1]
{
    \begin{mdframed}[%
        nobreak=true,%
        topline=false,%
        bottomline=false,%
        rightline=false%
        ]
    \section*{#1}
}
{
    \end{mdframed}
}
\newenvironment{flashcard}{}{}
\newenvironment{question}{\paragraph{Question}}{\vspace{5pt}}


\begin{document}
\begin{zettel}{Isometrie}

    Isometrien erhalten (in Anf"uhrungszeichen) alles was uns geometrisch interessiert.
\begin{flashcard}
    \begin{question}
    Was ist die Defintion einer Isometrie (mit Metrik)
    \end{question}
    \begin{definition}[Isometrie mit Metrik]
        \[
            F: (X,d_X) \longrightarrow (Y, d_Y)   \tx{ heisst Isometrie, falls } d_Y(f(x), f(y)) = d_X (x,y) \forall x,y
        .\]
    \end{definition}
\end{flashcard}

\begin{flashcard}
    \begin{question}
    Was ist die Defintion einer Isometrie (mit Skalarprodukt)
    \end{question}
    \begin{definition}[Isometrie mit Skalarprodukt]
        $(V_1, \langle .\rangle_1 ) \cdot  (V_2, \langle .\rangle_2) \quad \mathbb{K}\tx{-R"aume}$  
        Eine Isometrie ist eine bijektive lineare Abbildung: $F: V_1 \to V_2$ , sodass $ \langle F(v),F(v')\rangle_2 = \langle v,v'\rangle_1$ $\forall v,v' \in  V_1$ 
    \end{definition}
\end{flashcard}

\begin{lemma}["aquivalente Defintion]

$F$-linear, $F: (V_1, \langle .\rangle_1) \to (V_2, \langle .\rangle_2)$ ist eine Isometrie $\iff F$ schickt ONB zu ONB.\\
Das ist "aquivalent dazu, dass \emph{eine} Orthogonalbasis existiert, sodass $\dots$ 
\end{lemma}


\begin{remark}
$v,v' \in  V_1$ $F$ - Isometrie
\begin{enumerate}
    \item $ \norm{v}_1 = \norm{F(v)}_2$ 
    \item $d_1 (v,v') = \norm{v-v'}_1 = \norm{F(v)-F(v')}_2 = d_2 (F(v),F(v'))$ 
\end{enumerate}
\end{remark}

\begin{remark}
\[
    V_1 \stackrel{F}{\to} V_2 \stackrel{G}{\to} V_3
.\]
\begin{enumerate}
    \item $G \circ  F$ ist eine Isometrie
    \item $F^{-1}: V_2 \to  V_1$ ist eine Isometrie
\end{enumerate}

\end{remark}

\begin{corollary}
    $\mathcal{B}$-Basis von V, 
\[
    \Phi_B : (\mathbb{K}^n, \langle.,.\rangle_{std})  \to (V, \langle .\rangle) \tx{ - Isometrie }  \iff \mathcal{B} \tx{ ist eine ONB} 
.\]
\end{corollary}

\begin{corollary}
$\dim V_1 = \dim V_2 \implies  (V_2, \langle .\rangle_1), (V_2, \langle .\rangle_2)$  sind isometrisch.
\end{corollary}

\begin{theorem}[7.5]
    Sei $\mathcal{B}$-eine ONB von $(V, \langle .\rangle_v)$ und $F \in  \End(V)$, dann ist folgendes "aquivalent
    \begin{enumerate}
        \item $F$ ist eine Isometrie
        \item die Spalten von $M_B^B (F) $ sind eine ONB von $( \mathbb{K}^n, \langle .\rangle_{std})$ 
    \end{enumerate}
\end{theorem}
\end{zettel}
\end{document}


