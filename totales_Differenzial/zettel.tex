%! TeX root = zettel.tex
\documentclass[class=article, crop=false]{standalone}
\usepackage[subpreambles=true]{standalone}
% basics
\usepackage[utf8]{inputenc}
\usepackage[ngerman]{babel}
\usepackage{amsmath,amssymb,amsfonts,amsthm}
\usepackage{thmtools}
\usepackage{mathtools} % prettier math
\usepackage{mathdots}
\usepackage{enumitem}
\usepackage{comment}

% frames
\usepackage[linewidth=1pt]{mdframed}
\usepackage{framed}
\usepackage{tcolorbox}
\definecolor{shadecolor}{rgb}{0.9,0.9,0.9}

% graphics
\usepackage{import}
\usepackage{xifthen}
\usepackage{pdfpages}

% amsthm config
\declaretheoremstyle[notebraces={[}{]},headpunct= ,]{custom}

\theoremstyle{custom}
\newtheorem*{theorem}{Theorem}
\newtheorem*{lemma}{Lemma}
\newtheorem*{corollary}{Corollary}

\theoremstyle{custom}
\newtheorem*{axiom}{Axiom}
\newtheorem*{definition}{Definition}
\newtheorem*{example}{Example}

% symbol shortcuts
\newcommand{\zz}{\mathrm{Z\kern-.4em\raise-0.5ex\hbox{Z}}}
\newcommand{\tx}[1]{\text{ #1 }}
\newcommand{\La}{\mathcal{L}}
\newcommand{\N}{\mathbb{N}}
\newcommand{\K}{\mathbb{K}}
\newcommand{\R}{\mathbb{R}}
\newcommand{\Q}{\mathbb{Q}}

% math operators
\DeclareMathOperator{\Mat}{Mat}
\DeclareMathOperator{\sgn}{sgn}
\DeclareMathOperator{\Eig}{Eig}
\DeclareMathOperator{\Image}{Im}
\DeclareMathOperator{\Hom}{Hom}
\DeclareMathOperator{\End}{End}
\DeclareMathOperator{\GL}{GL}

% misc
\setlength\parindent{0pt}

\makeatletter

% nice way to write sets
\newcommand{\set}[1]{\@ifnextchar\bgroup {\left\{#1\setwithcondition} {\{#1\}} }
\newcommand{\setwithcondition}[1]{\ \setseperator \ #1 \right\}}
\newcommand{\setseperator}{\middle|}

\makeatother

\newenvironment{zettel}[1]
{
    \begin{mdframed}[%
        nobreak=true,%
        topline=false,%
        bottomline=false,%
        rightline=false%
        ]
    \section*{#1}
}
{
    \end{mdframed}
}
\newenvironment{flashcard}{}{}
\newenvironment{question}{\paragraph{Question}}{\vspace{5pt}}


\begin{document}
\begin{zettel}{totales Differenzial}
\begin{flashcard}[]{}
	\begin{question}
		totales Diffenrezial
	\end{question}
	\begin{definition}[Differenzierbarkeit mit totalem Differenzial]
		Sei $U \subset  V$ eine offene Teilmenge. $(V,\norm{.}_V), (W,\norm{.})$ sind endlich dimensionale normierte $K$-Vektorr"aume. $f \from U\to W$-hei"st diffbar an $a \in  U \subset  V$, falls $\exists  K$-lineare Abbildung $L \from V\to W$-mit $h \mapsto L (h)$ sodass
		\[
			\lim_{h \to 0} \frac{ \norm{f (a+h)- f (a)- L (h)}}{ \norm{h}} =0
		.\]

		Das hei"st $\forall\varepsilon>0 \exists\delta>0$, sodass $\forall h \in V$ mit $\norm{h}_V < \delta$ gilt
		\[
			\norm{ f (a +h) - f (a)- L (h)}_W < \varepsilon \cdot  \norm{h}_V
		.\]

		$D_a := L $ ist das totale Differenzial von $f$  an $a \in  U$
	\end{definition}
\end{flashcard}

\begin{remark}
	Ist $f$ diffbar, so ist das Totaldifferenzial eindeutig.
\end{remark}

\begin{remark}
	Das Totaldifferenzial einer linearen Abbildung ist die Abbildung selbst.
\end{remark}

\begin{remark}
	\begin{enumerate}
		\item Man kann Differenzierbarkeit auch f"ur $\infty$ dimensionale normirte Vektorr"aume definieren. Da nicht alle linearen Abbildungen zwischen $\infty$ dimensionalen Vektorr"aumen stetig sind, muss dies zus"atzlich verlangt werden.
		\item $K$-Linearit"at ist wichtig. Beispielsweise ist $f: C \to \R^2, z \mapsto \bar{z}$ $\R $ linear, allerdings nicht $C $-linear
	\end{enumerate}
\end{remark}

\end{zettel}
\end{document}

