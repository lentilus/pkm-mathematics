%! TeX root = zettel.tex
\documentclass[class=article, crop=false]{standalone}
\usepackage[subpreambles=true]{standalone}
% basics
\usepackage[utf8]{inputenc}
\usepackage[ngerman]{babel}
\usepackage{amsmath,amssymb,amsfonts,amsthm}
\usepackage{thmtools}
\usepackage{mathtools} % prettier math
\usepackage{mathdots}
\usepackage{enumitem}
\usepackage{comment}

% frames
\usepackage[linewidth=1pt]{mdframed}
\usepackage{framed}
\usepackage{tcolorbox}
\definecolor{shadecolor}{rgb}{0.9,0.9,0.9}

% graphics
\usepackage{import}
\usepackage{xifthen}
\usepackage{pdfpages}

% amsthm config
\declaretheoremstyle[notebraces={[}{]},headpunct= ,]{custom}

\theoremstyle{custom}
\newtheorem*{theorem}{Theorem}
\newtheorem*{lemma}{Lemma}
\newtheorem*{corollary}{Corollary}

\theoremstyle{custom}
\newtheorem*{axiom}{Axiom}
\newtheorem*{definition}{Definition}
\newtheorem*{example}{Example}

% symbol shortcuts
\newcommand{\zz}{\mathrm{Z\kern-.4em\raise-0.5ex\hbox{Z}}}
\newcommand{\tx}[1]{\text{ #1 }}
\newcommand{\La}{\mathcal{L}}
\newcommand{\N}{\mathbb{N}}
\newcommand{\K}{\mathbb{K}}
\newcommand{\R}{\mathbb{R}}
\newcommand{\Q}{\mathbb{Q}}

% math operators
\DeclareMathOperator{\Mat}{Mat}
\DeclareMathOperator{\sgn}{sgn}
\DeclareMathOperator{\Eig}{Eig}
\DeclareMathOperator{\Image}{Im}
\DeclareMathOperator{\Hom}{Hom}
\DeclareMathOperator{\End}{End}
\DeclareMathOperator{\GL}{GL}

% misc
\setlength\parindent{0pt}

\makeatletter

% nice way to write sets
\newcommand{\set}[1]{\@ifnextchar\bgroup {\left\{#1\setwithcondition} {\{#1\}} }
\newcommand{\setwithcondition}[1]{\ \setseperator \ #1 \right\}}
\newcommand{\setseperator}{\middle|}

\makeatother

\newenvironment{zettel}[1]
{
    \begin{mdframed}[%
        nobreak=true,%
        topline=false,%
        bottomline=false,%
        rightline=false%
        ]
    \section*{#1}
}
{
    \end{mdframed}
}
\newenvironment{flashcard}{}{}
\newenvironment{question}{\paragraph{Question}}{\vspace{5pt}}


\begin{document}
\begin{zettel}{zusammenhaengend}
\begin{flashcard}[]{}
	\begin{definition}[zusammenh"angende Mengen]
		$(x,\tau)$ ist nicht zus"ammenh"angend, falls
		\[
			\exists  A,B \subset X \quad A,B  \tx{ abgeschlossen, offen und disjunkt } (A \cap B =  \emptyset), X =  A \cup B \tx{ und } A,B \neq \emptyset
		.\]
		\begin{remark}
			$A$ - offen $\implies $ $X \setminus A =  B$ - abgeschlossen, D.h. sind $a$ , $B$ offen, so sind sie auch abgeschlossen.
		\end{remark}
	\end{definition}
\end{flashcard}
\begin{definition}[zusammenhaengende top. R"aume]
	$(X,\tau^x)$ topologischer Raum $Y \subset X$ - Teilmenge. Dann ist $Y$ nicht zusammenh"angend falls $ (Y,\tau^x)$ nicht zusammenh"angend.

	Hier ist $\tau^y =  \set{U \cap Y}{U \in  \tau^x}$ - eine induzierte Topologie

	In anderen Worten $Y$ ist nicht zusammenhaengend, falls

	\begin{align*}
		 & \exists U_1 , U_2 \in  \tau^x \text{ s.d. } Y = ( Y \cap U_1 ) \cup ( Y \cap U_2), ( Y \cap U_1 ) \cap ( Y \cap U_2) = \emptyset \\
		 & \tx{ und } Y \cap Ui \neq  \emptyset \tx{ f"ur } i = 1,2
	\end{align*}
\end{definition}
\end{zettel}

\begin{lemma}
	$Y \subset \R $ ist zusammenh"angend $\iff $ $Y$ ist ein Intervall
\end{lemma}

\begin{theorem}
	Sei $(X,\tau^x)$-zusammenhaengende top. R"aum, $ (Y,\tau^x)$ top. Raum , $f \from X\to Y $ - stetig. Dann ist $f(x)$ zusammenhaengend
\end{theorem}
\end{document}
