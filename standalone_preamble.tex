\usepackage[utf8]{inputenc}
\usepackage[ngerman]{babel}
\usepackage{amsmath,amssymb,amsfonts,amsthm}
\usepackage{faktor}
\usepackage{thmtools}
\usepackage{mathtools} % prettier math
\usepackage{mathdots}
\usepackage{enumitem}
\usepackage{comment}
\usepackage{etoolbox}

\usepackage{currfile}
\usepackage{subfiles}

\usepackage{xparse} % expl3 stuff

% frames
\usepackage[linewidth=1pt]{mdframed}
\usepackage{framed}
\usepackage{tcolorbox}
\definecolor{shadecolor}{rgb}{0.9,0.9,0.9}

% graphics
\usepackage{import}
\usepackage{xifthen}
\usepackage{pdfpages}
\usepackage{transparent}

% amsthm config
\declaretheoremstyle[notebraces={[}{]},headpunct=\newline,]{custom}
\theoremstyle{custom}
\newtheorem*{theorem}{Theorem}
\newtheorem*{lemma}{Lemma}
\newtheorem*{corollary}{Corollary}

\theoremstyle{custom}
\newtheorem*{axiom}{Axiom}
\newtheorem*{definition}{Definition}
\newtheorem*{example}{Example}
\newtheorem*{remark}{Remark}

% symbol shortcuts
\newcommand{\La}{\mathcal{L}}
\newcommand{\N}{\mathbb{N}}
\newcommand{\K}{\mathbb{K}}
\newcommand{\R}{\mathbb{R}}
\newcommand{\Q}{\mathbb{Q}}
\newcommand{\C}{\mathbb{C}}

% math operators
\DeclareMathOperator{\Mat}{Mat}
\DeclareMathOperator{\Spur}{Spur}
\DeclareMathOperator{\Rad}{Rad}
\DeclareMathOperator{\Hess}{Hess}
\DeclareMathOperator{\Sym}{Sym}
\DeclareMathOperator{\sgn}{sgn}
\DeclareMathOperator{\Eig}{Eig}
\DeclareMathOperator{\Image}{Im}
\DeclareMathOperator{\Iso}{Iso}
\DeclareMathOperator{\Isom}{Iso} % legacy
\DeclareMathOperator{\Isome}{Isom}
\DeclareMathOperator{\Hom}{Hom}
\DeclareMathOperator{\Bil}{Bil}
\DeclareMathOperator{\chr}{char}
\DeclareMathOperator{\End}{End}
\DeclareMathOperator{\Vol}{Vol}
\DeclareMathOperator{\GL}{GL}
\DeclareMathOperator{\HP}{HP}
\DeclareMathOperator{\rand}{rand}
\DeclareMathOperator{\ord}{ord}
\DeclareMathOperator{\grad}{grad}

% misc
\setlength\parindent{0pt}
\pagestyle{empty}

%%% Redefine existing commands
\let\oldforall\forall
\let\oldexists\exists
\renewcommand{\forall}{\ \oldforall}
\renewcommand{\exists}{\ \oldexists}
\renewcommand{\bar}{\overline}

%%% Macros
\makeatletter
% nice way to write sets
\newcommand{\set}[1]{\@ifnextchar\bgroup {\left\{#1\filteredset} {\{#1\}} }
\newcommand{\filteredset}[1]{\ \setseperator \ #1 \right\}}
\newcommand{\setseperator}{\middle|}
% derivatives
\newcommand\del[1]{\@ifnextchar /
{\@del{#1}}
{\@del/{#1}}}
\def\@del#1/#2{\frac{\partial #1}{\partial #2}}
\makeatother

\ExplSyntaxOn
\clist_new:N \l_feq_vector_clist
\NewDocumentCommand{\vect}{O{\\}mO{p}}{
\clist_set:Nn \l_feq_vector_clist {#2} % Set the list
\begin{#3matrix}
\clist_use:Nn \l_feq_vector_clist {#1} % show it with separator from #1 (\\)
\end{#3matrix}
}
\ExplSyntaxOff

\newcommand{\zz}{\mathrm{Z\kern-.4em\raise-0.5ex\hbox{Z}}}
\newcommand{\tx}[1]{\text{ #1 }}
\newcommand{\from}{\colon}
\newcommand{\norm}[1]{\left\|#1\right\|}
\newcommand{\constrain}[1]{_{\big{|_{#1}}}}

% \newenvironment{flashcard}{
% 	\begin{mdframed}[%
% 			backgroundcolor=black!15,%
% 			rightline=false,%
% 			topline=false,%
% 			bottomline=true,%
% 			leftline=false%
% 		]
% 		}{
% 	\end{mdframed}
% }
\newenvironment{flashcard}[2][none]{
	\begin{mdframed}[%
			backgroundcolor=black!15,%
			rightline=false,%
			topline=false,%
			bottomline=true,%
			leftline=false%
		]
		#2 \hfill \itshape #1\\
		}
		{
	\end{mdframed}
}

\newenvironment{question}{\par\itshape}{\par}
\newenvironment{zettel}[1]
{
	\begin{mdframed}[%
			nobreak=false,%
			topline=false,%
			bottomline=false,%
			rightline=false%
		]
		\section*{#1}
		}
		{
	\end{mdframed}
}

% no idea about the curly brace warning here...
\newcommand{\inkfig}[1]{
	\begingroup\centering%
	\def\svgwidth{\columnwidth}%
	\import{\currfiledir figures}{#1.pdf_tex}%
	\endgroup%
}

