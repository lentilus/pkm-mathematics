%! TeX root = zettel.tex
\documentclass[class=article, crop=false]{standalone}
\usepackage[subpreambles=true]{standalone}
% basics
\usepackage[utf8]{inputenc}
\usepackage[ngerman]{babel}
\usepackage{amsmath,amssymb,amsfonts,amsthm}
\usepackage{faktor}
\usepackage{thmtools}
\usepackage{mathtools} % prettier math
\usepackage{mathdots}
\usepackage{enumitem}
\usepackage{comment}
\usepackage{etoolbox}

\usepackage{currfile}
\usepackage{subfiles}

% frames
\usepackage[linewidth=1pt]{mdframed}
\usepackage{framed}
\usepackage{tcolorbox}
\definecolor{shadecolor}{rgb}{0.9,0.9,0.9}

% graphics
\usepackage{import}
\usepackage{xifthen}
\usepackage{pdfpages}
\usepackage{transparent}

% amsthm config
\declaretheoremstyle[notebraces={[}{]},headpunct=\newline,]{custom}
\theoremstyle{custom}
\newtheorem*{theorem}{Theorem}
\newtheorem*{lemma}{Lemma}
\newtheorem*{corollary}{Corollary}

\theoremstyle{custom}
\newtheorem*{axiom}{Axiom}
\newtheorem*{definition}{Definition}
\newtheorem*{example}{Example}
\newtheorem*{remark}{Remark}

% symbol shortcuts
\newcommand{\zz}{\mathrm{Z\kern-.4em\raise-0.5ex\hbox{Z}}}
\newcommand{\tx}[1]{\text{ #1 }}
\newcommand{\from}{\colon}
\newcommand{\La}{\mathcal{L}}
\newcommand{\N}{\mathbb{N}}
\newcommand{\K}{\mathbb{K}}
\newcommand{\R}{\mathbb{R}}
\newcommand{\Q}{\mathbb{Q}}
\newcommand{\C}{\mathbb{C}}

% math operators
\DeclareMathOperator{\Mat}{Mat}
\DeclareMathOperator{\sgn}{sgn}
\DeclareMathOperator{\Eig}{Eig}
\DeclareMathOperator{\Image}{Im}
\DeclareMathOperator{\Hom}{Hom}
\DeclareMathOperator{\End}{End}
\DeclareMathOperator{\GL}{GL}
\DeclareMathOperator{\HP}{HP}
\DeclareMathOperator{\rand}{rand}
\DeclareMathOperator{\ord}{ord}

% misc
\setlength\parindent{0pt}

% quantifiers
\let\oldforall\forall
\let\oldexists\exists
\renewcommand{\forall}{\ \oldforall}
\renewcommand{\exists}{\ \oldexists}

\makeatletter
    % nice way to write sets
    \newcommand{\set}[1]{\@ifnextchar\bgroup {\left\{#1\filteredset} {\{#1\}} }
    \newcommand{\filteredset}[1]{\ \setseperator \ #1 \right\}}
    \newcommand{\setseperator}{\middle|}
\makeatother
\newcommand{\norm}[1]{\left|\left|#1\right|\right|}

\newenvironment{flashcard}{}{}
\newenvironment{question}{\paragraph{Question}}{\vspace{5pt}}
\newenvironment{zettel}[1]
{
    \begin{mdframed}[%
        nobreak=true,%
        topline=false,%
        bottomline=false,%
        rightline=false%
        ]
    \section*{#1}
}
{
    \end{mdframed}
}
\newcommand{\inkfig}[1]{%
    % we cant use figure here because of boxes...
    \centering
    \def\svgwidth{\columnwidth}
    \import{\currfiledir figures}{#1.pdf_tex}
}


\begin{document}
\begin{zettel}{wegzusammenhaengend}
\begin{flashcard}[lfefcdxz]{wegzusammenhaengend}
	\begin{definition}
		$(X,\tau)$ ist wegzusammenhaengend falls
		\[
			\forall x_0,x_1 \in  X \exists \tx{ stetige } y \from [0,1] \to X \tx{ mit } y(i)=  x_i \tx{ f"ur } i =  0,1
		.\]
	\end{definition}
	\inkfig{wegzusammenhaengend}
\end{flashcard}
\begin{example}
	Ein Beispiel ist $S^2$
\end{example}
\begin{lemma}
	$X$ - wegzusammenhaengend $\implies $ $X$ - zusammenh"angend
\end{lemma}
\begin{lemma}
	Sei $U \subset \R^n$-offene Teilmenge. Dann $U$-zusammenh"angend $\iff $ $U$ wegzusammenhaengend
\end{lemma}
\begin{proof}
	\begin{enumerate}
		\item $\impliedby $ : - Folgt aus vorherigem Lemma.
		\item $\implies $ : Sei $x \in  U$-und $U_1 =\set{y \in  U}{\tx{ $x$-und $y$-kann man mit stetiger Kurve verbinden } }$ $U_1 \neq \emptyset $-, weil $x \in  U$-(konstante Kurve). $U_1$ ist offen. Tats"achlich sei $y \in  U_1$. Dann $\exists \varepsilon > 0 \tx{s.d.} B_{\varepsilon}(y) \subset  U$. Dann ist $B_{\varepsilon}(y) \subset  U_1$-weil $\forall y' \in  B_{\varepsilon}(y) \bar{y} = \begin{cases}
				      y (2t)           & t \in  [0,\frac{1}{2}] \\
				      y + (2t-1)(y'-y) & t \in  [\frac{1}{2},1]
			      \end{cases}$\\

		      $U \setminus U_1$ ist auch offen. Beweis: Sei $z \in  U \setminus  U_1 $, dann $\exists >0 \tx{s.d.} B_{\varepsilon}(z) \subset  U$. Ist $B_{\varepsilon}(z) \cap  U_1 \neq  \emptyset $, dann $\exists  z' \in  B_{\varepsilon}(z), z' \in  U_1$, so verbindet $y$ $x$ und $z$-und dann verbindet $\bar{\bar{y}} = \begin{cases}
				      y(2t)           & t \in [0,\frac{1}{2}]   \\
				      z'+(2t-1)(z-z') & t \in  [\frac{1}{2}, 1]
			      \end{cases}$\\

		      $\implies B_{\varepsilon}(z) \subset U \setminus U_1$ und $U \setminus U_1$ ist offen. $\implies $ $U = U_1 \cup (U \setminus  U_1)$ . Da $U$ zusammenh"angend und $U \neq  \emptyset $, so ist $U \setminus U_1 = \emptyset $ , also $U = U_1 \implies U$ ist wegzusammenhaengend.

	\end{enumerate}
\end{proof}
\end{zettel}
\end{document}

