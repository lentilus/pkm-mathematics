%! TeX root = zettel.tex
\documentclass[class=article, crop=false]{standalone}
\usepackage[subpreambles=true]{standalone}
% basics
\usepackage[utf8]{inputenc}
\usepackage[ngerman]{babel}
\usepackage{amsmath,amssymb,amsfonts,amsthm}
\usepackage{thmtools}
\usepackage{mathtools} % prettier math
\usepackage{mathdots}
\usepackage{enumitem}
\usepackage{comment}

% frames
\usepackage[linewidth=1pt]{mdframed}
\usepackage{framed}
\usepackage{tcolorbox}
\definecolor{shadecolor}{rgb}{0.9,0.9,0.9}

% graphics
\usepackage{import}
\usepackage{xifthen}
\usepackage{pdfpages}

% amsthm config
\declaretheoremstyle[notebraces={[}{]},headpunct= ,]{custom}

\theoremstyle{custom}
\newtheorem*{theorem}{Theorem}
\newtheorem*{lemma}{Lemma}
\newtheorem*{corollary}{Corollary}

\theoremstyle{custom}
\newtheorem*{axiom}{Axiom}
\newtheorem*{definition}{Definition}
\newtheorem*{example}{Example}

% symbol shortcuts
\newcommand{\zz}{\mathrm{Z\kern-.4em\raise-0.5ex\hbox{Z}}}
\newcommand{\tx}[1]{\text{ #1 }}
\newcommand{\La}{\mathcal{L}}
\newcommand{\N}{\mathbb{N}}
\newcommand{\K}{\mathbb{K}}
\newcommand{\R}{\mathbb{R}}
\newcommand{\Q}{\mathbb{Q}}

% math operators
\DeclareMathOperator{\Mat}{Mat}
\DeclareMathOperator{\sgn}{sgn}
\DeclareMathOperator{\Eig}{Eig}
\DeclareMathOperator{\Image}{Im}
\DeclareMathOperator{\Hom}{Hom}
\DeclareMathOperator{\End}{End}
\DeclareMathOperator{\GL}{GL}

% misc
\setlength\parindent{0pt}

\makeatletter

% nice way to write sets
\newcommand{\set}[1]{\@ifnextchar\bgroup {\left\{#1\setwithcondition} {\{#1\}} }
\newcommand{\setwithcondition}[1]{\ \setseperator \ #1 \right\}}
\newcommand{\setseperator}{\middle|}

\makeatother

\newenvironment{zettel}[1]
{
    \begin{mdframed}[%
        nobreak=true,%
        topline=false,%
        bottomline=false,%
        rightline=false%
        ]
    \section*{#1}
}
{
    \end{mdframed}
}
\newenvironment{flashcard}{}{}
\newenvironment{question}{\paragraph{Question}}{\vspace{5pt}}


\begin{document}
\begin{zettel}{Stetigkeit}
\begin{flashcard}
\begin{question}
    Definition Stetigkeit
\end{question}

%%% BEG DEFINITION: Stetigkeit %%%
\vspace*{-8pt}
\begin{definition}[Stetigkeit]
    $f:X \longrightarrow Y$ ist stetig an $x_0 \in  X$ , falls
\[
    \forall \varepsilon >  0 \exists  \delta  > 0 \text{ s.d. }  d (x_0 , x) < \varepsilon \implies \longrightarrow  d (f (x_0), f (x)) <  \varepsilon
.\]

    $f$ ist stetig, wenn wenn $f$ in jedem Punkt stetig ist.
    In anderen Worten
\[
    f (B_{\delta }(x_0) \subset B_{\varepsilon } (f (x_0)) )
.\]
\end{definition}
%%% END DEFINITION: Stetigkeit %%%
\end{flashcard}

\begin{definition}[Stetigkeit]
\begin{align*}
    f:X \longrightarrow Y \text{ ist stetig }  &\iff \forall \text{ offenen } U \subset Y \text{ ist } f^{-1} (U) \text{ offen } \\
                                               & \iff \forall  \text{ abgeschl. } D \subset Y \text{ ist } f^{-1} (D) \text{ agebschl. } )
\end{align*}
    Diese Definition ist "aquivalent zur $\varepsilon $,$\delta$-Definition, aber unter der Verwendung topologischer Argumente.
\end{definition}

\begin{flashcard}
\end{flashcard}
\end{zettel}
\end{document}
