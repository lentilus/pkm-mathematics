%! TeX root = zettel.tex
\documentclass[class=article, crop=false]{standalone}
\usepackage[subpreambles=true]{standalone}
% basics
\usepackage[utf8]{inputenc}
\usepackage[ngerman]{babel}
\usepackage{amsmath,amssymb,amsfonts,amsthm}
\usepackage{faktor}
\usepackage{thmtools}
\usepackage{mathtools} % prettier math
\usepackage{mathdots}
\usepackage{enumitem}
\usepackage{comment}
\usepackage{etoolbox}

\usepackage{currfile}
\usepackage{subfiles}

% frames
\usepackage[linewidth=1pt]{mdframed}
\usepackage{framed}
\usepackage{tcolorbox}
\definecolor{shadecolor}{rgb}{0.9,0.9,0.9}

% graphics
\usepackage{import}
\usepackage{xifthen}
\usepackage{pdfpages}
\usepackage{transparent}

% amsthm config
\declaretheoremstyle[notebraces={[}{]},headpunct=\newline,]{custom}
\theoremstyle{custom}
\newtheorem*{theorem}{Theorem}
\newtheorem*{lemma}{Lemma}
\newtheorem*{corollary}{Corollary}

\theoremstyle{custom}
\newtheorem*{axiom}{Axiom}
\newtheorem*{definition}{Definition}
\newtheorem*{example}{Example}
\newtheorem*{remark}{Remark}

% symbol shortcuts
\newcommand{\zz}{\mathrm{Z\kern-.4em\raise-0.5ex\hbox{Z}}}
\newcommand{\tx}[1]{\text{ #1 }}
\newcommand{\from}{\colon}
\newcommand{\La}{\mathcal{L}}
\newcommand{\N}{\mathbb{N}}
\newcommand{\K}{\mathbb{K}}
\newcommand{\R}{\mathbb{R}}
\newcommand{\Q}{\mathbb{Q}}
\newcommand{\C}{\mathbb{C}}

% math operators
\DeclareMathOperator{\Mat}{Mat}
\DeclareMathOperator{\sgn}{sgn}
\DeclareMathOperator{\Eig}{Eig}
\DeclareMathOperator{\Image}{Im}
\DeclareMathOperator{\Hom}{Hom}
\DeclareMathOperator{\End}{End}
\DeclareMathOperator{\GL}{GL}
\DeclareMathOperator{\HP}{HP}
\DeclareMathOperator{\rand}{rand}
\DeclareMathOperator{\ord}{ord}

% misc
\setlength\parindent{0pt}

% quantifiers
\let\oldforall\forall
\let\oldexists\exists
\renewcommand{\forall}{\ \oldforall}
\renewcommand{\exists}{\ \oldexists}

\makeatletter
    % nice way to write sets
    \newcommand{\set}[1]{\@ifnextchar\bgroup {\left\{#1\filteredset} {\{#1\}} }
    \newcommand{\filteredset}[1]{\ \setseperator \ #1 \right\}}
    \newcommand{\setseperator}{\middle|}
\makeatother
\newcommand{\norm}[1]{\left|\left|#1\right|\right|}

\newenvironment{flashcard}{}{}
\newenvironment{question}{\paragraph{Question}}{\vspace{5pt}}
\newenvironment{zettel}[1]
{
    \begin{mdframed}[%
        nobreak=true,%
        topline=false,%
        bottomline=false,%
        rightline=false%
        ]
    \section*{#1}
}
{
    \end{mdframed}
}
\newcommand{\inkfig}[1]{%
    % we cant use figure here because of boxes...
    \centering
    \def\svgwidth{\columnwidth}
    \import{\currfiledir figures}{#1.pdf_tex}
}


\begin{document}
\begin{zettel}{Stetigkeit}
\begin{flashcard}[z7o3jrfv]{Stetigkeit $\varepsilon, \delta$ }
	Stetigkeit - $\varepsilon, \delta $

	\begin{definition}[Stetigkeit]
		$f:X \longrightarrow Y$ ist stetig an $x_0 \in  X$ , falls
		\[
			\forall \varepsilon >  0 \exists  \delta  > 0 \text{ s.d. }  d (x_0 , x) < \delta \implies (f (x_0), f (x)) <  \varepsilon
		.\]

		$f$ ist stetig, wenn wenn $f$ in jedem Punkt stetig ist.
		In anderen Worten
		\[
			f (B_{\delta }(x_0)) \subset B_{\varepsilon } (f (x_0)) ).
		\]
	\end{definition}
	%%% END DEFINITION: Stetigkeit %%%
\end{flashcard}

\begin{flashcard}[9wbgtes1]{Stetigkeit - topologisch}
	\begin{definition}[Stetigkeit]
		\begin{align*}
			f:X \longrightarrow Y \text{ ist stetig } & \iff \forall \text{ offenen } U \subset Y \text{ ist } f^{-1} (U) \text{ offen }            \\
			                                          & ( \iff \forall  \text{ abgeschl. } D \subset Y \text{ ist } f^{-1} (D) \text{ agebschl. } )
		\end{align*}
		Diese Definition ist "aquivalent zur $\varepsilon $,$\delta$-Definition, aber unter der Verwendung topologischer Argumente.
	\end{definition}
\end{flashcard}

\begin{definition}[Stetigkeit]
	$(X, \tau_Y), (Y, \tau_y)$  - top. R"aume. $f:X \longrightarrow Y$ heisst stetig, falls
	\[
		\forall U \in  \tau_Y \tx{ ist } f^{-1} (U) \in  \tau_X.
	.\]
\end{definition}

"aquivalent ist
\[
	\overline{f(A)} \supset f (\overline{A})
.\]

\end{zettel}
\end{document}
