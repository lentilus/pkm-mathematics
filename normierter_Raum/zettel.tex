%! TeX root = zettel.tex
\documentclass[class=article, crop=false]{standalone}
\usepackage[subpreambles=true]{standalone}
% basics
\usepackage[utf8]{inputenc}
\usepackage[ngerman]{babel}
\usepackage{amsmath,amssymb,amsfonts,amsthm}
\usepackage{faktor}
\usepackage{thmtools}
\usepackage{mathtools} % prettier math
\usepackage{mathdots}
\usepackage{enumitem}
\usepackage{comment}
\usepackage{etoolbox}

\usepackage{currfile}
\usepackage{subfiles}

% frames
\usepackage[linewidth=1pt]{mdframed}
\usepackage{framed}
\usepackage{tcolorbox}
\definecolor{shadecolor}{rgb}{0.9,0.9,0.9}

% graphics
\usepackage{import}
\usepackage{xifthen}
\usepackage{pdfpages}
\usepackage{transparent}

% amsthm config
\declaretheoremstyle[notebraces={[}{]},headpunct=\newline,]{custom}
\theoremstyle{custom}
\newtheorem*{theorem}{Theorem}
\newtheorem*{lemma}{Lemma}
\newtheorem*{corollary}{Corollary}

\theoremstyle{custom}
\newtheorem*{axiom}{Axiom}
\newtheorem*{definition}{Definition}
\newtheorem*{example}{Example}
\newtheorem*{remark}{Remark}

% symbol shortcuts
\newcommand{\zz}{\mathrm{Z\kern-.4em\raise-0.5ex\hbox{Z}}}
\newcommand{\tx}[1]{\text{ #1 }}
\newcommand{\from}{\colon}
\newcommand{\La}{\mathcal{L}}
\newcommand{\N}{\mathbb{N}}
\newcommand{\K}{\mathbb{K}}
\newcommand{\R}{\mathbb{R}}
\newcommand{\Q}{\mathbb{Q}}
\newcommand{\C}{\mathbb{C}}

% math operators
\DeclareMathOperator{\Mat}{Mat}
\DeclareMathOperator{\sgn}{sgn}
\DeclareMathOperator{\Eig}{Eig}
\DeclareMathOperator{\Image}{Im}
\DeclareMathOperator{\Hom}{Hom}
\DeclareMathOperator{\End}{End}
\DeclareMathOperator{\GL}{GL}
\DeclareMathOperator{\HP}{HP}
\DeclareMathOperator{\rand}{rand}
\DeclareMathOperator{\ord}{ord}

% misc
\setlength\parindent{0pt}

% quantifiers
\let\oldforall\forall
\let\oldexists\exists
\renewcommand{\forall}{\ \oldforall}
\renewcommand{\exists}{\ \oldexists}

\makeatletter
    % nice way to write sets
    \newcommand{\set}[1]{\@ifnextchar\bgroup {\left\{#1\filteredset} {\{#1\}} }
    \newcommand{\filteredset}[1]{\ \setseperator \ #1 \right\}}
    \newcommand{\setseperator}{\middle|}
\makeatother
\newcommand{\norm}[1]{\left|\left|#1\right|\right|}

\newenvironment{flashcard}{}{}
\newenvironment{question}{\paragraph{Question}}{\vspace{5pt}}
\newenvironment{zettel}[1]
{
    \begin{mdframed}[%
        nobreak=true,%
        topline=false,%
        bottomline=false,%
        rightline=false%
        ]
    \section*{#1}
}
{
    \end{mdframed}
}
\newcommand{\inkfig}[1]{%
    % we cant use figure here because of boxes...
    \centering
    \def\svgwidth{\columnwidth}
    \import{\currfiledir figures}{#1.pdf_tex}
}


\begin{document}
\begin{zettel}{normierter Raum}
\begin{flashcard}[]{}
	\begin{definition}[normierter Raum]
		Ein Vektorraum $V$ zusammen mit
		\[
			|| . ||: V \longrightarrow [0, \infty), v \mapsto  ||v||
		.\]
		falls folgene Axiome erf"ullt sind.
		\begin{enumerate}
			\item $||v|| = 0 \implies v = 0 $
			\item $||v + w || \leq  ||v|| + ||w||$
			\item $||\lambda \cdot v|| = |\lambda | \cdot ||v||$
		\end{enumerate}
	\end{definition}
	Ein normierter VR ist ein metr. Raum mit $ d_{ ||\ ||} (v,w) := ||v -w||$
\end{flashcard}

\begin{remark}
	Beispiele f"ur lustige normierte R"aume :)
	\[
		l_p (\mathbb{N}) := \set{ (x_1 )}{ ||(x_n)|| = \sqrt[p]{\sum_{i=0}^{\infty}
				x_i|^p}  } \text{ sind auch Banachr"aume }
	.\]

	\begin{example}[Hilbertraum?]
		$\R$-Vektorraum V,
		\[
			\langle \ , \ \rangle : V \times V \longrightarrow \R, \langle v,w\rangle \in  \mathbb{R}
		\] ,sodass
		\begin{enumerate}
			\item $ \langle v,w\rangle  =  \langle w,v\rangle $
			\item $ \langle v_1+\lambda v_2,w\rangle =  \langle v_1 ,w\rangle  + \lambda \langle v_2,w\rangle $
			\item $ \langle v,v\rangle \geq 0$ und $ \langle v,v\rangle = 0 \iff  v = 0$

		\end{enumerate}
		Man deiniert dann eine Norm $||v|| = \sqrt{ \langle v,v\rangle }$

	\end{example}
\end{remark}

\begin{definition}
	V-VR. Zwei Normen $||\ ||_1, ||\ ||_2 $ auf $V$ hei"sen "aquivalent, falls $\exists c, C \in  (0,\infty) $ sd
	$\forall  v \in  V : c \cdot  ||v||_1 \leq ||v||_2 \leq C \cdot ||v||_1$

	- hier fehlt ggf noch etwas. Nicht sicher...
\end{definition}
\end{zettel}
\end{document}
