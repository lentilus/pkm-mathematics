%! TeX root = zettel.tex
\documentclass[class=article, crop=false]{standalone}
\usepackage[subpreambles=true]{standalone}
% basics
\usepackage[utf8]{inputenc}
\usepackage[ngerman]{babel}
\usepackage{amsmath,amssymb,amsfonts,amsthm}
\usepackage{faktor}
\usepackage{thmtools}
\usepackage{mathtools} % prettier math
\usepackage{mathdots}
\usepackage{enumitem}
\usepackage{comment}
\usepackage{etoolbox}

\usepackage{currfile}
\usepackage{subfiles}

% frames
\usepackage[linewidth=1pt]{mdframed}
\usepackage{framed}
\usepackage{tcolorbox}
\definecolor{shadecolor}{rgb}{0.9,0.9,0.9}

% graphics
\usepackage{import}
\usepackage{xifthen}
\usepackage{pdfpages}
\usepackage{transparent}

% amsthm config
\declaretheoremstyle[notebraces={[}{]},headpunct=\newline,]{custom}
\theoremstyle{custom}
\newtheorem*{theorem}{Theorem}
\newtheorem*{lemma}{Lemma}
\newtheorem*{corollary}{Corollary}

\theoremstyle{custom}
\newtheorem*{axiom}{Axiom}
\newtheorem*{definition}{Definition}
\newtheorem*{example}{Example}
\newtheorem*{remark}{Remark}

% symbol shortcuts
\newcommand{\zz}{\mathrm{Z\kern-.4em\raise-0.5ex\hbox{Z}}}
\newcommand{\tx}[1]{\text{ #1 }}
\newcommand{\from}{\colon}
\newcommand{\La}{\mathcal{L}}
\newcommand{\N}{\mathbb{N}}
\newcommand{\K}{\mathbb{K}}
\newcommand{\R}{\mathbb{R}}
\newcommand{\Q}{\mathbb{Q}}
\newcommand{\C}{\mathbb{C}}

% math operators
\DeclareMathOperator{\Mat}{Mat}
\DeclareMathOperator{\sgn}{sgn}
\DeclareMathOperator{\Eig}{Eig}
\DeclareMathOperator{\Image}{Im}
\DeclareMathOperator{\Hom}{Hom}
\DeclareMathOperator{\End}{End}
\DeclareMathOperator{\GL}{GL}
\DeclareMathOperator{\HP}{HP}
\DeclareMathOperator{\rand}{rand}
\DeclareMathOperator{\ord}{ord}

% misc
\setlength\parindent{0pt}

% quantifiers
\let\oldforall\forall
\let\oldexists\exists
\renewcommand{\forall}{\ \oldforall}
\renewcommand{\exists}{\ \oldexists}

\makeatletter
    % nice way to write sets
    \newcommand{\set}[1]{\@ifnextchar\bgroup {\left\{#1\filteredset} {\{#1\}} }
    \newcommand{\filteredset}[1]{\ \setseperator \ #1 \right\}}
    \newcommand{\setseperator}{\middle|}
\makeatother
\newcommand{\norm}[1]{\left|\left|#1\right|\right|}

\newenvironment{flashcard}{}{}
\newenvironment{question}{\paragraph{Question}}{\vspace{5pt}}
\newenvironment{zettel}[1]
{
    \begin{mdframed}[%
        nobreak=true,%
        topline=false,%
        bottomline=false,%
        rightline=false%
        ]
    \section*{#1}
}
{
    \end{mdframed}
}
\newcommand{\inkfig}[1]{%
    % we cant use figure here because of boxes...
    \centering
    \def\svgwidth{\columnwidth}
    \import{\currfiledir figures}{#1.pdf_tex}
}


\begin{document}
\begin{zettel}{Homoeomorphismus}
\begin{flashcard}
    nicht zu verwechseln mit Homomorphismus!!!
\begin{definition}[Homoeomorphismus]
   Eine bijektive Abbldung $g:X \longrightarrow Y$  heisst ein Homomorphismus, falls $f^{-1} $  auch stetig ist.
   \begin{remark}
      Wir sehen Homoeomorphismen als eine Art von "Aquivalenz von top. R"aumen. 
   \end{remark}
\end{definition}

\begin{example}
    \begin{enumerate}
    \item \[
        f: \mathbb{R}^n \setminus \set{\emptyset} \longrightarrow  \mathbb{R}^n , f(x) =  \frac{x}{ ||x||}
    .\]
    $ ||f(x)|| = ||\frac{x}{||x||}|| = \frac{1}{ ||x||} ||x|| = 1 $ 
    Das sind alle Vektoren mit L"ange 1.
     \item \[
         f: B_1 (0) \longrightarrow \mathbb{R}^n
     .\]
     \[
         f(x) =  \frac{x}{1 - ||x||} 
     .\]
     $f$ bildet einen Strahl in sich selbst. $\implies f$ ist bijektiv. Keine Isometrie

     \item Inversion $I: \mathbb{R}^n \setminus \set{0} \longrightarrow  \R^n \setminus 0$  \quad $I(x) = \frac{x}{ ||x||^2 } = \frac{1}{ ||x||}  $ Bildet auch einen Strahl in sich selbst. $I$ - Homoeomorphie
    \end{enumerate}
\end{example}

\begin{example}[Stereographische Projektion]
    Formel in n"achster Vorlesung. Hilft Sphere zu parametrisieren.
\end{example}

\end{flashcard}
\end{zettel}
\end{document}


