%! TeX root = zettel.tex
\documentclass[class=article, crop=false]{standalone}
\usepackage[subpreambles=true]{standalone}
% basics
\usepackage[utf8]{inputenc}
\usepackage[ngerman]{babel}
\usepackage{amsmath,amssymb,amsfonts,amsthm}
\usepackage{faktor}
\usepackage{thmtools}
\usepackage{mathtools} % prettier math
\usepackage{mathdots}
\usepackage{enumitem}
\usepackage{comment}
\usepackage{etoolbox}

\usepackage{currfile}
\usepackage{subfiles}

% frames
\usepackage[linewidth=1pt]{mdframed}
\usepackage{framed}
\usepackage{tcolorbox}
\definecolor{shadecolor}{rgb}{0.9,0.9,0.9}

% graphics
\usepackage{import}
\usepackage{xifthen}
\usepackage{pdfpages}
\usepackage{transparent}

% amsthm config
\declaretheoremstyle[notebraces={[}{]},headpunct=\newline,]{custom}
\theoremstyle{custom}
\newtheorem*{theorem}{Theorem}
\newtheorem*{lemma}{Lemma}
\newtheorem*{corollary}{Corollary}

\theoremstyle{custom}
\newtheorem*{axiom}{Axiom}
\newtheorem*{definition}{Definition}
\newtheorem*{example}{Example}
\newtheorem*{remark}{Remark}

% symbol shortcuts
\newcommand{\zz}{\mathrm{Z\kern-.4em\raise-0.5ex\hbox{Z}}}
\newcommand{\tx}[1]{\text{ #1 }}
\newcommand{\from}{\colon}
\newcommand{\La}{\mathcal{L}}
\newcommand{\N}{\mathbb{N}}
\newcommand{\K}{\mathbb{K}}
\newcommand{\R}{\mathbb{R}}
\newcommand{\Q}{\mathbb{Q}}
\newcommand{\C}{\mathbb{C}}

% math operators
\DeclareMathOperator{\Mat}{Mat}
\DeclareMathOperator{\sgn}{sgn}
\DeclareMathOperator{\Eig}{Eig}
\DeclareMathOperator{\Image}{Im}
\DeclareMathOperator{\Hom}{Hom}
\DeclareMathOperator{\End}{End}
\DeclareMathOperator{\GL}{GL}
\DeclareMathOperator{\HP}{HP}
\DeclareMathOperator{\rand}{rand}
\DeclareMathOperator{\ord}{ord}

% misc
\setlength\parindent{0pt}

% quantifiers
\let\oldforall\forall
\let\oldexists\exists
\renewcommand{\forall}{\ \oldforall}
\renewcommand{\exists}{\ \oldexists}

\makeatletter
    % nice way to write sets
    \newcommand{\set}[1]{\@ifnextchar\bgroup {\left\{#1\filteredset} {\{#1\}} }
    \newcommand{\filteredset}[1]{\ \setseperator \ #1 \right\}}
    \newcommand{\setseperator}{\middle|}
\makeatother
\newcommand{\norm}[1]{\left|\left|#1\right|\right|}

\newenvironment{flashcard}{}{}
\newenvironment{question}{\paragraph{Question}}{\vspace{5pt}}
\newenvironment{zettel}[1]
{
    \begin{mdframed}[%
        nobreak=true,%
        topline=false,%
        bottomline=false,%
        rightline=false%
        ]
    \section*{#1}
}
{
    \end{mdframed}
}
\newcommand{\inkfig}[1]{%
    % we cant use figure here because of boxes...
    \centering
    \def\svgwidth{\columnwidth}
    \import{\currfiledir figures}{#1.pdf_tex}
}


\begin{document}
\begin{zettel}{elementare Menge}
\begin{flashcard}[fbnseont]{elementare Menge}
	\begin{definition}
		Eine Elementare Menge $E$ ist eine Menge von der Form
		\[
			E = \bigcup_{s = 1}^k Q_s
		.\]
		Wobei $Q_s$ ein Quader ist.
	\end{definition}
\end{flashcard}

\begin{lemma}[Eigenschaften elementarer Mengen]
	Seien $E$ , $F \subset  \R ^n$ elementar.
	\begin{enumerate}
		\item $E \cup  F, E \cap F, E \setminus F, E \triangle F$ sind Elementar.
		\item $\forall x \in  \R^n $ ist $x + E := \set{x + y}{y \in  E}$ elementar.
		\item Alle elementaren $E \subset  \R^n $ sind von der Form $E = \bigsqcup_{s = 1 }^m Q_s$
		\item Falls $\bigsqcup_{s = 1 }^{m'} Q'_s = E = \bigsqcup_{s = 1 }^{m'} Q'_s \implies \sum_{s = 1}^{m'} | Q_s|_n = \sum_{s = 1}^{m} | Q_s|_n$
		\item $m(E) \geq 0, m(\emptyset) = 0$
		\item $m (E \sqcup F) = m(E) + m(F)$
		\item $m(Q) = |Q|$
		\item $E \subset F \implies  m(E) \leq  m(F)$
		\item $ m (E \cup  F) \leq  m(E)+ m(F)$
		\item $m ( x + E) = m(E)$
		\item $E_i \subset \R^n , i = 1,2$ elementare Mengen $\implies m_{n1 + n_2}(E_1 \times  E_2) = m_{ n_1 }(E_1) \cdot  m_{n_2} $
	\end{enumerate}
\end{lemma}

\begin{corollary}
	Elementare Mengen sind Jordanmessbar.
	\[
		m^J(E)= m(E)
	.\]
\end{corollary}
\end{zettel}
\end{document}

