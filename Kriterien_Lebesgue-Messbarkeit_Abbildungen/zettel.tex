%! TeX root = zettel.tex
\documentclass[class=article, crop=false]{standalone}
\usepackage[subpreambles=true]{standalone}
% basics
\usepackage[utf8]{inputenc}
\usepackage[ngerman]{babel}
\usepackage{amsmath,amssymb,amsfonts,amsthm}
\usepackage{faktor}
\usepackage{thmtools}
\usepackage{mathtools} % prettier math
\usepackage{mathdots}
\usepackage{enumitem}
\usepackage{comment}
\usepackage{etoolbox}

\usepackage{currfile}
\usepackage{subfiles}

% frames
\usepackage[linewidth=1pt]{mdframed}
\usepackage{framed}
\usepackage{tcolorbox}
\definecolor{shadecolor}{rgb}{0.9,0.9,0.9}

% graphics
\usepackage{import}
\usepackage{xifthen}
\usepackage{pdfpages}
\usepackage{transparent}

% amsthm config
\declaretheoremstyle[notebraces={[}{]},headpunct=\newline,]{custom}
\theoremstyle{custom}
\newtheorem*{theorem}{Theorem}
\newtheorem*{lemma}{Lemma}
\newtheorem*{corollary}{Corollary}

\theoremstyle{custom}
\newtheorem*{axiom}{Axiom}
\newtheorem*{definition}{Definition}
\newtheorem*{example}{Example}
\newtheorem*{remark}{Remark}

% symbol shortcuts
\newcommand{\zz}{\mathrm{Z\kern-.4em\raise-0.5ex\hbox{Z}}}
\newcommand{\tx}[1]{\text{ #1 }}
\newcommand{\from}{\colon}
\newcommand{\La}{\mathcal{L}}
\newcommand{\N}{\mathbb{N}}
\newcommand{\K}{\mathbb{K}}
\newcommand{\R}{\mathbb{R}}
\newcommand{\Q}{\mathbb{Q}}
\newcommand{\C}{\mathbb{C}}

% math operators
\DeclareMathOperator{\Mat}{Mat}
\DeclareMathOperator{\sgn}{sgn}
\DeclareMathOperator{\Eig}{Eig}
\DeclareMathOperator{\Image}{Im}
\DeclareMathOperator{\Hom}{Hom}
\DeclareMathOperator{\End}{End}
\DeclareMathOperator{\GL}{GL}
\DeclareMathOperator{\HP}{HP}
\DeclareMathOperator{\rand}{rand}
\DeclareMathOperator{\ord}{ord}

% misc
\setlength\parindent{0pt}

% quantifiers
\let\oldforall\forall
\let\oldexists\exists
\renewcommand{\forall}{\ \oldforall}
\renewcommand{\exists}{\ \oldexists}

\makeatletter
    % nice way to write sets
    \newcommand{\set}[1]{\@ifnextchar\bgroup {\left\{#1\filteredset} {\{#1\}} }
    \newcommand{\filteredset}[1]{\ \setseperator \ #1 \right\}}
    \newcommand{\setseperator}{\middle|}
\makeatother
\newcommand{\norm}[1]{\left|\left|#1\right|\right|}

\newenvironment{flashcard}{}{}
\newenvironment{question}{\paragraph{Question}}{\vspace{5pt}}
\newenvironment{zettel}[1]
{
    \begin{mdframed}[%
        nobreak=true,%
        topline=false,%
        bottomline=false,%
        rightline=false%
        ]
    \section*{#1}
}
{
    \end{mdframed}
}
\newcommand{\inkfig}[1]{%
    % we cant use figure here because of boxes...
    \centering
    \def\svgwidth{\columnwidth}
    \import{\currfiledir figures}{#1.pdf_tex}
}


\begin{document}
\begin{zettel}{Kriterien Lebesgue-Messbarkeit Abbildungen}
\begin{flashcard}[mp59nol3]{Kriterien Lebesgue-Messbarkeit einer Abbildung}
	$\Omega \in  \La_{\R^n}$. $f: \Omega \to  \R^m $ l-messbar $\iff \forall \mathring{Q} \subset  \R^m: f^{-1}(\mathring{Q}) \in  \La _{\R^n}$.
\end{flashcard}

\begin{flashcard}[jwxrd2q0]{Kriterien Lebesgue-Messbarkeit Abbildung mit mehreren Komponenten}
	$\Omega \in  \La_{\R^n} $ und $f: \Omega \to  \R ^m$
	\[
		f(x) = \vect{f_1(x), \vdots, f_m(x)} \tx{l-messbar}  \iff f_i \tx{l-messbar}  \forall i
	.\]
\end{flashcard}

\begin{flashcard}[o3r1hpqc]{lebesgue-Messbarkeit komposition}
	Sei $\Omega \stackrel{f}{\to}  \R^m \stackrel{g}{\to}  \R^k$, $f$ - l-messbar, $g$ stetig. $\implies$ $g \circ f$ l-messbar.
\end{flashcard}

\begin{flashcard}[n7ex401o]{
		Sei $f,g: \Omega \to  \R $  l-messbar. Dann sind folgende l-messbar:
	}
	\begin{corollary}[einige l-messbare Abbildungen]
		Sei $f,g: \Omega \to \R $  l-messbar. Dann sind folgende l-messbar:
		\begin{enumerate}
			\item $|f |$
			\item $\max \set{f, 0}$
			\item $\min \set{f, 0}$
			\item $f + g$, $f - g$
			\item $\max \set{f, g}$
			\item $\min \set{f, g}$
		\end{enumerate}
	\end{corollary}
\end{flashcard}

\begin{flashcard}[774xx3ih]{Lebesgue-Messbarkeit $\Omega \to  \hat{\R } $ }
	$\Omega \in  \lmable$, $f: \Omega \to  \hat{\R }$ falls $f^{-1} ([0, \infty])$ ist l-messbar.
\end{flashcard}

\begin{flashcard}[gdavk7f2]{l-Messbarkeit Supremum, Infimum}
	$\Omega \in \lmable$ und $f_n: \Omega \to  \hat{\R }$ l-messbar $\forall n \in  N$ dann
	\[
		\sup_n f_n, \quad, \inf_n f_n, \quad \inf_N \sup_{n \geq N} f_n, \quad \sup_N \inf_{n \geq N} f_n \quad \tx{ l-messbar}
	.\]
\end{flashcard}

\begin{lemma}[Eigenschaften]
	TODO
\end{lemma}

\end{zettel}
\end{document}

