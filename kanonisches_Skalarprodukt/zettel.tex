%! TeX root = zettel.tex
\documentclass[class=article, crop=false]{standalone}
\usepackage[subpreambles=true]{standalone}
% basics
\usepackage[utf8]{inputenc}
\usepackage[ngerman]{babel}
\usepackage{amsmath,amssymb,amsfonts,amsthm}
\usepackage{faktor}
\usepackage{thmtools}
\usepackage{mathtools} % prettier math
\usepackage{mathdots}
\usepackage{enumitem}
\usepackage{comment}
\usepackage{etoolbox}

\usepackage{currfile}
\usepackage{subfiles}

% frames
\usepackage[linewidth=1pt]{mdframed}
\usepackage{framed}
\usepackage{tcolorbox}
\definecolor{shadecolor}{rgb}{0.9,0.9,0.9}

% graphics
\usepackage{import}
\usepackage{xifthen}
\usepackage{pdfpages}
\usepackage{transparent}

% amsthm config
\declaretheoremstyle[notebraces={[}{]},headpunct=\newline,]{custom}
\theoremstyle{custom}
\newtheorem*{theorem}{Theorem}
\newtheorem*{lemma}{Lemma}
\newtheorem*{corollary}{Corollary}

\theoremstyle{custom}
\newtheorem*{axiom}{Axiom}
\newtheorem*{definition}{Definition}
\newtheorem*{example}{Example}
\newtheorem*{remark}{Remark}

% symbol shortcuts
\newcommand{\zz}{\mathrm{Z\kern-.4em\raise-0.5ex\hbox{Z}}}
\newcommand{\tx}[1]{\text{ #1 }}
\newcommand{\from}{\colon}
\newcommand{\La}{\mathcal{L}}
\newcommand{\N}{\mathbb{N}}
\newcommand{\K}{\mathbb{K}}
\newcommand{\R}{\mathbb{R}}
\newcommand{\Q}{\mathbb{Q}}
\newcommand{\C}{\mathbb{C}}

% math operators
\DeclareMathOperator{\Mat}{Mat}
\DeclareMathOperator{\sgn}{sgn}
\DeclareMathOperator{\Eig}{Eig}
\DeclareMathOperator{\Image}{Im}
\DeclareMathOperator{\Hom}{Hom}
\DeclareMathOperator{\End}{End}
\DeclareMathOperator{\GL}{GL}
\DeclareMathOperator{\HP}{HP}
\DeclareMathOperator{\rand}{rand}
\DeclareMathOperator{\ord}{ord}

% misc
\setlength\parindent{0pt}

% quantifiers
\let\oldforall\forall
\let\oldexists\exists
\renewcommand{\forall}{\ \oldforall}
\renewcommand{\exists}{\ \oldexists}

\makeatletter
    % nice way to write sets
    \newcommand{\set}[1]{\@ifnextchar\bgroup {\left\{#1\filteredset} {\{#1\}} }
    \newcommand{\filteredset}[1]{\ \setseperator \ #1 \right\}}
    \newcommand{\setseperator}{\middle|}
\makeatother
\newcommand{\norm}[1]{\left|\left|#1\right|\right|}

\newenvironment{flashcard}{}{}
\newenvironment{question}{\paragraph{Question}}{\vspace{5pt}}
\newenvironment{zettel}[1]
{
    \begin{mdframed}[%
        nobreak=true,%
        topline=false,%
        bottomline=false,%
        rightline=false%
        ]
    \section*{#1}
}
{
    \end{mdframed}
}
\newcommand{\inkfig}[1]{%
    % we cant use figure here because of boxes...
    \centering
    \def\svgwidth{\columnwidth}
    \import{\currfiledir figures}{#1.pdf_tex}
}


\begin{document}
\begin{zettel}{kanonisches Skalarprodukt}
\begin{flashcard}[7rvb3gk4]{kanonisches Skalarprodukt}
	\begin{definition}[kanonisches Skalarprodukt $\R^n$ ]
		das Kanonische Skalarprodukt auf $\mathbb{R}^n $  ist die Abbildung
		\begin{align*}
			\langle .,.\rangle : \mathbb{R}^n \times \mathbb{R}^n                                                                  & \longrightarrow \mathbb{R}                                                                                                                           \\
			\begin{pmatrix}x_1\\ \vdots\\ x_n\end{pmatrix}, \begin{pmatrix}y_2\\ \vdots\\ y_n\end{pmatrix} & \mapsto \sum_{i = 1}^{k} x_i  y_i  = (x_1 , \dots , x_n ) \cdot \begin{pmatrix}y_1\\ \dots\\ y_n\end{pmatrix}  = \bar{y}^t \cdot \bar{y}
		\end{align*}
	\end{definition}
\end{flashcard}

% \begin{enumerate}
% 	\item $ \langle \bar{x},\bar{y}_1+\bar{y_2}\rangle  = \langle \bar{x},\bar{y_1}\rangle + \langle x_1,y_2\rangle $\\
% 	      $ \langle x,\lambda y \rangle = \lambda \langle x,y\rangle $
% 	\item $ \langle x,y\rangle = \langle y,x\rangle $
% 	\item $ \langle x,x\rangle \geq 0 $ und $ \langle x,x\rangle  = 0 \iff  x = 0 \in  \mathbb{R}^n$
% \end{enumerate}

\begin{flashcard}[40po2j06]{}
	\begin{definition}[kanonisches Skalarprodukt in Complexen VR]
		siehe Skript vom 19.04
		\[
			\langle \cdot,\cdot\rangle : \mathbb{C} \times \mathbb{C} \longrightarrow  \mathbb{C}, \begin{pmatrix}v_1 \\ \vdots \\ v_n\end{pmatrix}, \begin{pmatrix}w_1 \\ \vdots \\ w_n\end{pmatrix} \mapsto \sum_{i = 1}^{n} v_i \bar{w_i}
		.\]
		Wo $ \bar{w_i} $ das komplex konjugierte von $w_i $ ist.
	\end{definition}
\end{flashcard}
\end{zettel}
\end{document}

