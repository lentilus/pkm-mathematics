%! TeX root = zettel.tex
\documentclass[class=article, crop=false]{standalone}
\usepackage[subpreambles=true]{standalone}
% basics
\usepackage[utf8]{inputenc}
\usepackage[ngerman]{babel}
\usepackage{amsmath,amssymb,amsfonts,amsthm}
\usepackage{thmtools}
\usepackage{mathtools} % prettier math
\usepackage{mathdots}
\usepackage{enumitem}
\usepackage{comment}

% frames
\usepackage[linewidth=1pt]{mdframed}
\usepackage{framed}
\usepackage{tcolorbox}
\definecolor{shadecolor}{rgb}{0.9,0.9,0.9}

% graphics
\usepackage{import}
\usepackage{xifthen}
\usepackage{pdfpages}

% amsthm config
\declaretheoremstyle[notebraces={[}{]},headpunct= ,]{custom}

\theoremstyle{custom}
\newtheorem*{theorem}{Theorem}
\newtheorem*{lemma}{Lemma}
\newtheorem*{corollary}{Corollary}

\theoremstyle{custom}
\newtheorem*{axiom}{Axiom}
\newtheorem*{definition}{Definition}
\newtheorem*{example}{Example}

% symbol shortcuts
\newcommand{\zz}{\mathrm{Z\kern-.4em\raise-0.5ex\hbox{Z}}}
\newcommand{\tx}[1]{\text{ #1 }}
\newcommand{\La}{\mathcal{L}}
\newcommand{\N}{\mathbb{N}}
\newcommand{\K}{\mathbb{K}}
\newcommand{\R}{\mathbb{R}}
\newcommand{\Q}{\mathbb{Q}}

% math operators
\DeclareMathOperator{\Mat}{Mat}
\DeclareMathOperator{\sgn}{sgn}
\DeclareMathOperator{\Eig}{Eig}
\DeclareMathOperator{\Image}{Im}
\DeclareMathOperator{\Hom}{Hom}
\DeclareMathOperator{\End}{End}
\DeclareMathOperator{\GL}{GL}

% misc
\setlength\parindent{0pt}

\makeatletter

% nice way to write sets
\newcommand{\set}[1]{\@ifnextchar\bgroup {\left\{#1\setwithcondition} {\{#1\}} }
\newcommand{\setwithcondition}[1]{\ \setseperator \ #1 \right\}}
\newcommand{\setseperator}{\middle|}

\makeatother

\newenvironment{zettel}[1]
{
    \begin{mdframed}[%
        nobreak=true,%
        topline=false,%
        bottomline=false,%
        rightline=false%
        ]
    \section*{#1}
}
{
    \end{mdframed}
}
\newenvironment{flashcard}{}{}
\newenvironment{question}{\paragraph{Question}}{\vspace{5pt}}


\begin{document}
\begin{zettel}{Integral aus Mass}
\begin{flashcard}[jaecw7pt]{Integral von nicht-negativen Funktionen}
	\begin{definition}[Integral von nicht-negativen Funktionen]
		$f: X \to  [0,\infty]$  messbar, dann
		\[
			\int_{X} fd\mu := \sup \set{e \int \int_{X} gd\mu}{g \leq  f, g \tx{einfach} }
		.\]
	\end{definition}
\end{flashcard}
\begin{lemma}[Eigenschaften]
	TODO
\end{lemma}

\begin{flashcard}[aqs7rf0k]{Annaeherung messbarer Funktion durch einfache}
	\begin{lemma}
		$f:X \to  [0,\infty] \implies \exists$ Folge einfacher Funktionen $s_1 \leq \dots \leq s_k \leq \dots$ mit
		\[
			f(x) = \lim_{k \to \infty} s_k(x).
		.\]
	\end{lemma}
\end{flashcard}

\begin{flashcard}[r1396eph]{monotone Konvergenz Integral}
	\begin{definition}
		$f_n : X \to  [0,\infty]$ aufsteigende Folge messbarer Funktionen. Dann
		\[
			\int_{X} \sup f_n d\mu = \sup \int_{X} f_n d\mu
		.\]
	\end{definition}
\end{flashcard}

\begin{flashcard}[d1o0mf6y]{Additivitaet Integral}
	$f,g : X \to  [0,\infty]$ messbar. Dann
	\[
		\int_{X}(f + g) d\mu = \int_{X} fd\mu + \int_{X} gd\mu
	.\]
\end{flashcard}

\begin{flashcard}[1qexfp0s]{absolut integrierbar}
	\begin{definition}[absolut Integrierbar]
		$f: X \to  \hat{\R }$ messbar (bez"uglich $B$ auf $X$ , $B_{\R}$ auf $\R $) heisst (absolut) integrierbar falls
		\[
			\int_{X} f_{\pm }d\mu < \infty \iff \int_{X} |f| d\mu < \infty
		.\]
	\end{definition}
	Man kann sich mit Zusaetzen etwas aehnliches fuer $\C $ basteln.
\end{flashcard}

\begin{flashcard}[c5nqxbd1]{Majorisierte Konvergenz (allgemein)}
	\begin{theorem}[Majorisierte Konvergenz]
		$f_n :X \to  \hat{\R }$ Folge mesbarer Funktionen. $F \in L'(X,B,\mu)$ mit $ |f_n(x) | < |F(x)| \quad \forall x$.
		Dann
		\[
			\lim_{n \to \infty} \int_{X} f_n d\mu = \int_{X} \left(\lim_{n \to \infty} f_n \right) d\mu
		.\]
	\end{theorem}
\end{flashcard}
\end{zettel}
\end{document}

