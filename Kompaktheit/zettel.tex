%! TeX root = zettel.tex
\documentclass[class=article, crop=false]{standalone}
\usepackage[subpreambles=true]{standalone}
% basics
\usepackage[utf8]{inputenc}
\usepackage[ngerman]{babel}
\usepackage{amsmath,amssymb,amsfonts,amsthm}
\usepackage{thmtools}
\usepackage{mathtools} % prettier math
\usepackage{mathdots}
\usepackage{enumitem}
\usepackage{comment}

% frames
\usepackage[linewidth=1pt]{mdframed}
\usepackage{framed}
\usepackage{tcolorbox}
\definecolor{shadecolor}{rgb}{0.9,0.9,0.9}

% graphics
\usepackage{import}
\usepackage{xifthen}
\usepackage{pdfpages}

% amsthm config
\declaretheoremstyle[notebraces={[}{]},headpunct= ,]{custom}

\theoremstyle{custom}
\newtheorem*{theorem}{Theorem}
\newtheorem*{lemma}{Lemma}
\newtheorem*{corollary}{Corollary}

\theoremstyle{custom}
\newtheorem*{axiom}{Axiom}
\newtheorem*{definition}{Definition}
\newtheorem*{example}{Example}

% symbol shortcuts
\newcommand{\zz}{\mathrm{Z\kern-.4em\raise-0.5ex\hbox{Z}}}
\newcommand{\tx}[1]{\text{ #1 }}
\newcommand{\La}{\mathcal{L}}
\newcommand{\N}{\mathbb{N}}
\newcommand{\K}{\mathbb{K}}
\newcommand{\R}{\mathbb{R}}
\newcommand{\Q}{\mathbb{Q}}

% math operators
\DeclareMathOperator{\Mat}{Mat}
\DeclareMathOperator{\sgn}{sgn}
\DeclareMathOperator{\Eig}{Eig}
\DeclareMathOperator{\Image}{Im}
\DeclareMathOperator{\Hom}{Hom}
\DeclareMathOperator{\End}{End}
\DeclareMathOperator{\GL}{GL}

% misc
\setlength\parindent{0pt}

\makeatletter

% nice way to write sets
\newcommand{\set}[1]{\@ifnextchar\bgroup {\left\{#1\setwithcondition} {\{#1\}} }
\newcommand{\setwithcondition}[1]{\ \setseperator \ #1 \right\}}
\newcommand{\setseperator}{\middle|}

\makeatother

\newenvironment{zettel}[1]
{
    \begin{mdframed}[%
        nobreak=true,%
        topline=false,%
        bottomline=false,%
        rightline=false%
        ]
    \section*{#1}
}
{
    \end{mdframed}
}
\newenvironment{flashcard}{}{}
\newenvironment{question}{\paragraph{Question}}{\vspace{5pt}}


\begin{document}
\begin{zettel}{Kompaktheit}
\begin{definition}[Kompaktheit (Heine Borelsche Eigenschaft)]
    Eine Teilmenge $K \subset  X$ hei"st kompakt, falls $\forall $ offene "Uberdeckung $\mathcal{U} =\set{U_{\alpha}\in  \tau}{\alpha \in I}$ von $K$ 
\[
    K = \bigcup_{\alpha \in  I} U_\alpha
.\]
gibt es eine endliche Teil"uberdeckung $\exists \alpha, \dots, \alpha_n \in  I$ sd $K \subset  U_{\alpha_{1}} \cup \dots \cup U_{\alpha_n}$ 
\end{definition}

\begin{remark}
    Eigenschaften
\begin{itemize}
    \item $[a,b] \subset \R$ ist kompakt,  $-\infty < a,b < \infty $
    \item $F$ - abgeschl. K-kompakt $\implies F \cap K$ ist kompakt
    \item Unedliche $E \subset K $-kompakt  $\implies \exists x \in  HP (E) \cap K$ 
    \item $K$ - kompakt $\implies $ $K$ ist abgeschl.
\end{itemize}
\end{remark}

\begin{theorem}
$K$ - kompakt $\iff  $  $K$ abgeschlossen und f"ur jede Folge $(x_n )$ sd $\set{x_n } \subset K$ eine konvgt. Teilfolge existiert.
\begin{proof}
    Unter Verwendung Lebequescher Zahl.
\end{proof}
\end{theorem}

\end{zettel}
\end{document}
