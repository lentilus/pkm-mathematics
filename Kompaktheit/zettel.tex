%! TeX root = zettel.tex
\documentclass[class=article, crop=false]{standalone}
\usepackage[subpreambles=true]{standalone}
% basics
\usepackage[utf8]{inputenc}
\usepackage[ngerman]{babel}
\usepackage{amsmath,amssymb,amsfonts,amsthm}
\usepackage{thmtools}
\usepackage{mathtools} % prettier math
\usepackage{mathdots}
\usepackage{enumitem}
\usepackage{comment}

% frames
\usepackage[linewidth=1pt]{mdframed}
\usepackage{framed}
\usepackage{tcolorbox}
\definecolor{shadecolor}{rgb}{0.9,0.9,0.9}

% graphics
\usepackage{import}
\usepackage{xifthen}
\usepackage{pdfpages}

% amsthm config
\declaretheoremstyle[notebraces={[}{]},headpunct= ,]{custom}

\theoremstyle{custom}
\newtheorem*{theorem}{Theorem}
\newtheorem*{lemma}{Lemma}
\newtheorem*{corollary}{Corollary}

\theoremstyle{custom}
\newtheorem*{axiom}{Axiom}
\newtheorem*{definition}{Definition}
\newtheorem*{example}{Example}

% symbol shortcuts
\newcommand{\zz}{\mathrm{Z\kern-.4em\raise-0.5ex\hbox{Z}}}
\newcommand{\tx}[1]{\text{ #1 }}
\newcommand{\La}{\mathcal{L}}
\newcommand{\N}{\mathbb{N}}
\newcommand{\K}{\mathbb{K}}
\newcommand{\R}{\mathbb{R}}
\newcommand{\Q}{\mathbb{Q}}

% math operators
\DeclareMathOperator{\Mat}{Mat}
\DeclareMathOperator{\sgn}{sgn}
\DeclareMathOperator{\Eig}{Eig}
\DeclareMathOperator{\Image}{Im}
\DeclareMathOperator{\Hom}{Hom}
\DeclareMathOperator{\End}{End}
\DeclareMathOperator{\GL}{GL}

% misc
\setlength\parindent{0pt}

\makeatletter

% nice way to write sets
\newcommand{\set}[1]{\@ifnextchar\bgroup {\left\{#1\setwithcondition} {\{#1\}} }
\newcommand{\setwithcondition}[1]{\ \setseperator \ #1 \right\}}
\newcommand{\setseperator}{\middle|}

\makeatother

\newenvironment{zettel}[1]
{
    \begin{mdframed}[%
        nobreak=true,%
        topline=false,%
        bottomline=false,%
        rightline=false%
        ]
    \section*{#1}
}
{
    \end{mdframed}
}
\newenvironment{flashcard}{}{}
\newenvironment{question}{\paragraph{Question}}{\vspace{5pt}}


\begin{document}
\begin{zettel}{Kompaktheit}
\begin{definition}[Kompaktheit (Heine Borelsche Eigenschaft)]

	Sei $X$-top Raum. Eine Teilmenge $K \subset X$ hei"st kompakt, falls $\forall $
	offene "Uberdeckung $\mathcal{U} =\set{U_{\alpha}\in \tau}{\alpha \in I}$ von
	$K$
	\[
		K = \bigcup_{\alpha \in  I} U_\alpha
	.\]
	gibt es eine endliche Teil"uberdeckung $\exists \alpha, \dots, \alpha_n \in I$
	sd $K \subset U_{\alpha_{1}} \cup \dots \cup U_{\alpha_n}$
\end{definition}

\begin{remark}
	Eigenschaften
	\begin{itemize}
		\item $[a,b] \subset \R$ ist kompakt,  $-\infty < a,b < \infty $
		\item $F$ - abgeschl. K-kompakt $\implies F \cap K$ ist kompakt
		\item Unedliche $E \subset K $-kompakt $\implies \exists x \in HP (E) \cap K$
		\item $K$ - kompakt $\implies $ $K$ ist abgeschl.
	\end{itemize}
\end{remark}

\begin{theorem}

	sa $K$ - kompakt $\iff $ $K$ abgeschlossen und f"ur jede Folge $(x_n )$ sd
	$\set{x_n } \subset K$ eine konvgt. Teilfolge existiert.

	\begin{proof}
		Unter Verwendung Lebequescher Zahl.
	\end{proof}
\end{theorem}
\begin{theorem}
	$(V,\tau)$-top Raum
	\begin{enumerate}
		\item K-kompakte, A-abgeschlossene Teilmengen von $X$ $\implies $ $K \cap A$ kompakt.
		\item kompakte Teilmengen, Hausdorffsche top R"aume sind abgeschlossen.
		\item $\prod_{i=1}^n [a_n,b_i] = \set{x \in  \R ^{n}}{ a_i \leq x_i \leq  b_i, i=1,\dots,n}$
		      ist kompakt. (Quader in $n$ Dimensionen)
		\item $M \subset  \R ^n$ ist kompakt $\iff $ $M$ abgeschlossen und beschr"ankt.
	\end{enumerate}
\end{theorem}

\begin{remark}
	Da metr. R"aume hausdorffsch, gilt 2. f"ur alle metr R"aume
\end{remark}
\begin{proof}
	$1$ bis $4$ in VL bewiesen.
	\begin{enumerate}
		\item Sei $\mathcal{U} = \set{U_{\alpha}}{\alpha \in I}$-offene "Uberdeckung von $A
			      \cap K$. Dann ist $\mathcal{U_{1}}= \mathcal{U}\cup \set{X \setminus A}$-offene
		      "Uberdeckung von $K$. Da $K$-tkompakt, so gilt es eine endliche
		      Teil"uberdeckung von $\mathcal{U}'$. ObdA $U_{\alpha_1}, \dots, U_{\alpha_n}, $
		      $,X \setminus A$ von $K: K \subset U_{\alpha} \cup \dots \cup U_{\alpha_{n}}
			      \cup X \setminus A$. Dann aber $K \cap A \subset U_{\alpha_{1}} \cup \dots \cup
			      \dots$-d.h. $\mathcal{U}$-besitzt eine endliche Teil"uberdeckung von $K \cap
			      A$-und so ist $K \cap A $ kompakt.
		\item (nicht mitgeschrieben)
		\item Bewiesen in $1$ dim im ersten Semester. $[a,b] \subset \R $ ist kompakt. Dies
		      basiert auf: $I_1 \supset I_2 \supset \dots \supset I_k \tx{ Familie
				      abgeschlossener Intervalle } $. Dann ist $\bigcap_{k=1}^{\infty} I_k \neq
			      \emptyset \tx{ dh } \exists x \in \bigcap_{k=1}^{\infty} I_k \quad (\tx{ zB. }
			      x =\sup(\min I_k))$

		      "Ahnlich f"ur eine Form geschachtelter Quader
		      \[
			      Q_1 \supset Q_2 \supset \dots Q_n
		      .\]\\
		      $\quad\vdots$ \\
		      (nicht zuende geschrieben)

	\end{enumerate}
\end{proof}
\end{zettel}
\end{document}
