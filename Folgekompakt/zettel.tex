%! TeX root = zettel.tex
\documentclass[class=article, crop=false]{standalone}
\usepackage[subpreambles=true]{standalone}
% basics
\usepackage[utf8]{inputenc}
\usepackage[ngerman]{babel}
\usepackage{amsmath,amssymb,amsfonts,amsthm}
\usepackage{faktor}
\usepackage{thmtools}
\usepackage{mathtools} % prettier math
\usepackage{mathdots}
\usepackage{enumitem}
\usepackage{comment}
\usepackage{etoolbox}

\usepackage{currfile}
\usepackage{subfiles}

% frames
\usepackage[linewidth=1pt]{mdframed}
\usepackage{framed}
\usepackage{tcolorbox}
\definecolor{shadecolor}{rgb}{0.9,0.9,0.9}

% graphics
\usepackage{import}
\usepackage{xifthen}
\usepackage{pdfpages}
\usepackage{transparent}

% amsthm config
\declaretheoremstyle[notebraces={[}{]},headpunct=\newline,]{custom}
\theoremstyle{custom}
\newtheorem*{theorem}{Theorem}
\newtheorem*{lemma}{Lemma}
\newtheorem*{corollary}{Corollary}

\theoremstyle{custom}
\newtheorem*{axiom}{Axiom}
\newtheorem*{definition}{Definition}
\newtheorem*{example}{Example}
\newtheorem*{remark}{Remark}

% symbol shortcuts
\newcommand{\zz}{\mathrm{Z\kern-.4em\raise-0.5ex\hbox{Z}}}
\newcommand{\tx}[1]{\text{ #1 }}
\newcommand{\from}{\colon}
\newcommand{\La}{\mathcal{L}}
\newcommand{\N}{\mathbb{N}}
\newcommand{\K}{\mathbb{K}}
\newcommand{\R}{\mathbb{R}}
\newcommand{\Q}{\mathbb{Q}}
\newcommand{\C}{\mathbb{C}}

% math operators
\DeclareMathOperator{\Mat}{Mat}
\DeclareMathOperator{\sgn}{sgn}
\DeclareMathOperator{\Eig}{Eig}
\DeclareMathOperator{\Image}{Im}
\DeclareMathOperator{\Hom}{Hom}
\DeclareMathOperator{\End}{End}
\DeclareMathOperator{\GL}{GL}
\DeclareMathOperator{\HP}{HP}
\DeclareMathOperator{\rand}{rand}
\DeclareMathOperator{\ord}{ord}

% misc
\setlength\parindent{0pt}

% quantifiers
\let\oldforall\forall
\let\oldexists\exists
\renewcommand{\forall}{\ \oldforall}
\renewcommand{\exists}{\ \oldexists}

\makeatletter
    % nice way to write sets
    \newcommand{\set}[1]{\@ifnextchar\bgroup {\left\{#1\filteredset} {\{#1\}} }
    \newcommand{\filteredset}[1]{\ \setseperator \ #1 \right\}}
    \newcommand{\setseperator}{\middle|}
\makeatother
\newcommand{\norm}[1]{\left|\left|#1\right|\right|}

\newenvironment{flashcard}{}{}
\newenvironment{question}{\paragraph{Question}}{\vspace{5pt}}
\newenvironment{zettel}[1]
{
    \begin{mdframed}[%
        nobreak=true,%
        topline=false,%
        bottomline=false,%
        rightline=false%
        ]
    \section*{#1}
}
{
    \end{mdframed}
}
\newcommand{\inkfig}[1]{%
    % we cant use figure here because of boxes...
    \centering
    \def\svgwidth{\columnwidth}
    \import{\currfiledir figures}{#1.pdf_tex}
}


\begin{document}
\begin{zettel}{Folgekompakt}
\begin{flashcard}
    \begin{definition}[folgekompakt]
        $K \subset  X$ hei"st folgekompakt, falls jede Folge aus $K$ eine konvergente Teilfolge mit Grenzwert in $K$ hat.
    \end{definition}
    \begin{theorem}
    $(V,\tau)$-top Raum
    \begin{enumerate}
        \item K-kompakte, Abgeschlossene Teilmengen von $X$ $\implies $ $K \cup A$ kompakt. 
        \item kompakte Teilmengen, Hausdorffsche top R"aume sind abgeschlossen.
    \item $\prod_{i=1}^n [a_n,b_i] = \set{x \in  \R ^{n}}{ a_i \leq x_i \leq  b_i, i=1,\dots,n}$ ist kompakt.
    \item $M \subset  \R ^n$ ist kompakt $\iff $ $M$ abgeschlossen und beschr"ankt.
    \end{enumerate}
    \end{theorem}

    \begin{remark}
    Da metr. R"aume hausdorffsch, gilt 2. f"ur alle metr R"aume
    \end{remark}
    \begin{proof}
        \begin{enumerate}
            \item Sei $\mathcal{U} = \set{U_{\alpha}}{\alpha \in  I}$-offene "Uberdeckung von $A \cap K$. Dann ist $\mathcal{U_{1}}= \mathcal{U}\cup \set{X \setminus  A}$-offene "Uberdeckung von $K$. Da $K$-tkompakt, so gilt es eine endliche Teil"uberdeckung von $\mathcal{U}'$. ObdA $U_{\alpha_1}, \dots, U_{\alpha_n}, $  $,X \setminus A$ von $K: K \subset  U_{\alpha} \cup \dots \cup U_{\alpha_{n}} \cup X \setminus A$. Dann aber $K \cap  A \subset U_{\alpha_{1}} \cup  \dots \cup \dots$-d.h. $\mathcal{U}$-besitzt eine endliche Teil"uberdeckung von $K \cap  A$-und so ist $K \cap  A $ kompakt.
            \item (nicht mitgeschrieben)
        \end{enumerate}
    \end{proof}

\end{flashcard}
\end{zettel}
\end{document}


