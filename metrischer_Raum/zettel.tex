%! TeX root = zettel.tex
\documentclass[class=article, crop=false]{standalone}
\usepackage[subpreambles=true]{standalone}
% basics
\usepackage[utf8]{inputenc}
\usepackage[ngerman]{babel}
\usepackage{amsmath,amssymb,amsfonts,amsthm}
\usepackage{faktor}
\usepackage{thmtools}
\usepackage{mathtools} % prettier math
\usepackage{mathdots}
\usepackage{enumitem}
\usepackage{comment}
\usepackage{etoolbox}

\usepackage{currfile}
\usepackage{subfiles}

% frames
\usepackage[linewidth=1pt]{mdframed}
\usepackage{framed}
\usepackage{tcolorbox}
\definecolor{shadecolor}{rgb}{0.9,0.9,0.9}

% graphics
\usepackage{import}
\usepackage{xifthen}
\usepackage{pdfpages}
\usepackage{transparent}

% amsthm config
\declaretheoremstyle[notebraces={[}{]},headpunct=\newline,]{custom}
\theoremstyle{custom}
\newtheorem*{theorem}{Theorem}
\newtheorem*{lemma}{Lemma}
\newtheorem*{corollary}{Corollary}

\theoremstyle{custom}
\newtheorem*{axiom}{Axiom}
\newtheorem*{definition}{Definition}
\newtheorem*{example}{Example}
\newtheorem*{remark}{Remark}

% symbol shortcuts
\newcommand{\zz}{\mathrm{Z\kern-.4em\raise-0.5ex\hbox{Z}}}
\newcommand{\tx}[1]{\text{ #1 }}
\newcommand{\from}{\colon}
\newcommand{\La}{\mathcal{L}}
\newcommand{\N}{\mathbb{N}}
\newcommand{\K}{\mathbb{K}}
\newcommand{\R}{\mathbb{R}}
\newcommand{\Q}{\mathbb{Q}}
\newcommand{\C}{\mathbb{C}}

% math operators
\DeclareMathOperator{\Mat}{Mat}
\DeclareMathOperator{\sgn}{sgn}
\DeclareMathOperator{\Eig}{Eig}
\DeclareMathOperator{\Image}{Im}
\DeclareMathOperator{\Hom}{Hom}
\DeclareMathOperator{\End}{End}
\DeclareMathOperator{\GL}{GL}
\DeclareMathOperator{\HP}{HP}
\DeclareMathOperator{\rand}{rand}
\DeclareMathOperator{\ord}{ord}

% misc
\setlength\parindent{0pt}

% quantifiers
\let\oldforall\forall
\let\oldexists\exists
\renewcommand{\forall}{\ \oldforall}
\renewcommand{\exists}{\ \oldexists}

\makeatletter
    % nice way to write sets
    \newcommand{\set}[1]{\@ifnextchar\bgroup {\left\{#1\filteredset} {\{#1\}} }
    \newcommand{\filteredset}[1]{\ \setseperator \ #1 \right\}}
    \newcommand{\setseperator}{\middle|}
\makeatother
\newcommand{\norm}[1]{\left|\left|#1\right|\right|}

\newenvironment{flashcard}{}{}
\newenvironment{question}{\paragraph{Question}}{\vspace{5pt}}
\newenvironment{zettel}[1]
{
    \begin{mdframed}[%
        nobreak=true,%
        topline=false,%
        bottomline=false,%
        rightline=false%
        ]
    \section*{#1}
}
{
    \end{mdframed}
}
\newcommand{\inkfig}[1]{%
    % we cant use figure here because of boxes...
    \centering
    \def\svgwidth{\columnwidth}
    \import{\currfiledir figures}{#1.pdf_tex}
}


\begin{document}
\begin{zettel}{metrischer Raum}
\begin{flashcard}
\begin{question}
    Wie ist die Definition eines metr. Raumes?
\end{question}
\begin{definition}[metrischer Raum]
    Sei $X$ - Menge, $d:X \times X \longrightarrow \R$ - Abbildung. Wir nennen $(X,d)$ metrischen Raum, wenn $\forall x,y,z \in X$ 
    \begin{enumerate}
        \item (Positive Definitheit) $d(x,y) \geq 0$ und $d(x,y)= 0 \implies x = y$ 
        \item (Symmetrie) $d(x,y) = d(y,x)$ 
        \item (Dreiecksunglecihung) $d(x,y) + d(y,z) \geq d(x,z)$ 
    \end{enumerate}
\end{definition}
\end{flashcard}

\begin{example}[Metriken]
Seien $ (X,d_X) , (Y,d_Y)  $ - metr. R"aume
\begin{itemize}
    \item $d_E ( (x_1,x_2) , (x_2,y_2)) := \sqrt{d_x (x_1,x_2)^2 + d_y (y_1,y_2)^2  }$ 
    \item $d_{max} (\dots\ \dots) := max \set{ d_X (x_1,x_2) , d_Y  (y_1,y_2) }$
    \item $d_{sum} (\dots\ \dots):= d_X (x_1,x_2) +  d_Y (y_1,y_2) $
\end{itemize}
\end{example}    

\begin{theorem}
   $(X,d)$ ein metr. Raum, so gilt $\forall a \in  \bar{Y} \exists $ Folge $(y_n )$ in $Y$ sd $\lim_{n \to \infty} y_n = a$ 
\end{theorem}

\end{zettel}
\end{document}
