%! TeX root = zettel.tex
\documentclass[class=article, crop=false]{standalone}
\usepackage[subpreambles=true]{standalone}
% basics
\usepackage[utf8]{inputenc}
\usepackage[ngerman]{babel}
\usepackage{amsmath,amssymb,amsfonts,amsthm}
\usepackage{thmtools}
\usepackage{mathtools} % prettier math
\usepackage{mathdots}
\usepackage{enumitem}
\usepackage{comment}

% frames
\usepackage[linewidth=1pt]{mdframed}
\usepackage{framed}
\usepackage{tcolorbox}
\definecolor{shadecolor}{rgb}{0.9,0.9,0.9}

% graphics
\usepackage{import}
\usepackage{xifthen}
\usepackage{pdfpages}

% amsthm config
\declaretheoremstyle[notebraces={[}{]},headpunct= ,]{custom}

\theoremstyle{custom}
\newtheorem*{theorem}{Theorem}
\newtheorem*{lemma}{Lemma}
\newtheorem*{corollary}{Corollary}

\theoremstyle{custom}
\newtheorem*{axiom}{Axiom}
\newtheorem*{definition}{Definition}
\newtheorem*{example}{Example}

% symbol shortcuts
\newcommand{\zz}{\mathrm{Z\kern-.4em\raise-0.5ex\hbox{Z}}}
\newcommand{\tx}[1]{\text{ #1 }}
\newcommand{\La}{\mathcal{L}}
\newcommand{\N}{\mathbb{N}}
\newcommand{\K}{\mathbb{K}}
\newcommand{\R}{\mathbb{R}}
\newcommand{\Q}{\mathbb{Q}}

% math operators
\DeclareMathOperator{\Mat}{Mat}
\DeclareMathOperator{\sgn}{sgn}
\DeclareMathOperator{\Eig}{Eig}
\DeclareMathOperator{\Image}{Im}
\DeclareMathOperator{\Hom}{Hom}
\DeclareMathOperator{\End}{End}
\DeclareMathOperator{\GL}{GL}

% misc
\setlength\parindent{0pt}

\makeatletter

% nice way to write sets
\newcommand{\set}[1]{\@ifnextchar\bgroup {\left\{#1\setwithcondition} {\{#1\}} }
\newcommand{\setwithcondition}[1]{\ \setseperator \ #1 \right\}}
\newcommand{\setseperator}{\middle|}

\makeatother

\newenvironment{zettel}[1]
{
    \begin{mdframed}[%
        nobreak=true,%
        topline=false,%
        bottomline=false,%
        rightline=false%
        ]
    \section*{#1}
}
{
    \end{mdframed}
}
\newenvironment{flashcard}{}{}
\newenvironment{question}{\paragraph{Question}}{\vspace{5pt}}


\begin{document}
\begin{zettel}{Satz von der lokalen Umkehrbarkeit}

\begin{flashcard}[svvujhnu]{Satz von der lokalen Umkehrbarkeit}
	\begin{theorem}[Satz von der lokalen Umkehrbarkeit]
		$V,W$ normierte Vektorräume, $\dim V,W < \infty$.  $U \subset V$
		\[
			U \stackrel{f}{\to} W,\quad f \in C'(U,W),\quad a \in U \text{ s.d } D_af: V \to W - \text{ Isomorphismus }
		.\]
		Dann $\exists $ offene $U_0 \subseteq U$ mit $a \in U_0$ s.d $f( U_0 ) $-offene Umgebung von $f( a ) $ und $f \constrain{U_0} $- Diffeomorphismus: $\exists $ inverses $g : f( U_0 ) \to U_0, g \in C'(f(U_0),V) $
	\end{theorem}
\end{flashcard}

\begin{proof}
	$\implies \Phi_y \cdot  \bar{B_{2\delta}(o)} \to  \bar{B_{2\delta}(o)}$ eine Kontraktion eines vollst"anideng metrichen Raumens. $\implies  \exists ! \tx{ Fixpunkt } x_y \in \bar{B_{2\delta}(o)}: \Phi_y(x_y) ee x_y \iff  f(x_y)= y \implies  B_{\delta}(o) \subset  f (B_{2\delta}(o)) $ und man definiert die Umkehrabbildung $g: B_{\delta}(o) \to  f^{-1}(B_{\delta}(o))$
	\[
		g(y) = x_y: f \circ  g = id_{B_{\delta}(o)} \quad g \circ  f = \dots
	.\]

	Zu zeigen:
	\begin{enumerate}
		\item $g$ - stetig
		\item $g$ -diffbar
		\item $D_g$ -stetig
	\end{enumerate}

	\begin{enumerate}
		\item $y_1, y_2 \in  B_{\delta}(o) \quad x_i = g(y_i) \quad i =1,2 \implies y_1 = f(x_i)$

		      $x_2 - x_1 = \Phi_o(x_2) - \Phi_o(x_1) + f(x_2) - f(x_1) \implies \norm{x_2 - x_1} \leq  \norm{\Phi_o(x_2) - \Phi_o(x_2)} + \norm{ f(x_2) - f(x_1)}$ $\leq  \frac{1}{2} \norm{x_2-x_1} + \norm{ f(x_2) - f(x_1)} \iff $
		      $\iff \norm{g(y_2) - g(y_1)} \leq  \norm{y_1 - y_2} \iff \norm{g(y_2) - g(y_1)} \leq 2\norm{y_1-y_2}$
		      $g$ ist Lipschtiz-stetig dh. $g$ ist stetig.

		\item $f$ ist diffbar an $D \in  U$ d.$h$ f"ur $R(h) = f(h) - D_o f(h) = f(h) - h$  gilt
		      \[
			      \frac{\norm{R(h)}}{\norm{h}} \stackrel{h \to  0}{\to} 0
		      .\]
		      F"ur $k \in  B_{\delta}(o)$ definiert man $id_V$ $S(k) = g(k) - k $
		      (Wir werden zeigen, dass $g$ diffbar an $o \in  B_{\delta}(o)$ mit $D_o g = id_V$ )\\
		      $k = f(h) = h + R(h), h = g(k) = k + S(k)$

		      $h = g(f(h)) = g(h + R(h)) = h + R(h) + S(h + R(h)) \implies R(h) = - S(h + R(h))$

		      Analog: $ R (g(k)) + S(k) \stackrel{\ast }{=} 0$. $\exists \delta_1 > 0 : \forall  h $
		      mit $\norm{h} \leq  \delta_1$ gilt $R(h) \leq  \frac{1}{2} \norm{h}$.
		      Aus $1.$ folgt $\exists  \delta_2 > 0 : \forall  k \tx{  mit }  \norm{k} < \delta_2$
		      gilt $\norm{g(k)} < \delta_1$ (Stetigkeit von $g$ )
		      $\ast $ : $ \norm{ S(k)} =  \norm{- R(g(k))} \leq  \frac{1}{2} \norm{g(k)}$
		      $ \implies  \norm{k} \geq  \norm{g(k) - S(k)} \geq  \frac{1}{2} \norm{g(k)}$
		      $\implies \norm{g(k)} \leq 2  \norm{k} \iff \frac{1}{\norm{k}} \stackrel{\ast \ast }{\leq} \frac{2}{ \norm{g(k)}}$

		      $\ast  \ast $ : nicht mehr mitgeshrieben. Wir folgern:
		      $\implies  \forall  b \in  B_{\delta}(o)$ ist $g$ an $b$ diffbar und $ (D_b g) = ( D_{g(b)} f )$
		\item $L \in  \Hom_\K(V,V) =: \stackrel{\tx{ normierter Raum } }{\End_ \K (V)} \supset Aut_k(V) = \set{L}{\exists L^{-1}}$ - offene Teilmenge ... nicht weiter mitgeshrieben
	\end{enumerate}

\end{proof}
\end{zettel}
\end{document}
